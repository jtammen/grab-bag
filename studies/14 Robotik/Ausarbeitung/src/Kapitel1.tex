\section{Was heißt Lernen?}
Die Psychologie bezeichnet jede Interaktion des Menschen mit seiner Umwelt - 
zum Beispiel die Reaktion auf Reize - als Verhalten. Jeder Mensch verfügt zu 
einem bestimmten Zeitpunkt seines Lebens über ein bestimmtes Repertoire von 
Verhaltensmöglichkeiten. Dieses Repertoire ändert sich im Laufe der Zeit. 
Liegen die Gründe für diese Änderung in äußeren Einwirkungen und nicht in der 
Biologie (Alterungsprozess), so spricht man von Lernen. Daraus folgt die 
folgende Definition:

\begin{quotation}
Lernen ist eine dauerhafte (im Gegensatz zu einer vorübergehenden) Änderung
des Verhaltens und von Verhaltenspotenzialen, die durch Übung (im Gegensatz etwa
zu Reifung, Prägung oder Krankheit) erfolgt.
\end{quotation}

\section{Was heißt Maschinelles Lernen?}
Der psychologischen Definition des Lernens stellen wir nun zwei Definitionen
des Maschinellen Lernens (engl. Machine Learning) gegenüber:

\begin{quotation}
Machine Learning is programming computers to optimize a performance
criterion using example data or past experience \cite{Alpaydin2004}.
\end{quotation}

\begin{quotation}
A computer program is said to learn from experience $E$ with respect to some 
class of tasks $T$ and performance measure $P$, if its performance at tasks in 
$T$, as measured by $P$, improves with experience $E$ \cite{Mitchell1997}.
\end{quotation}

\section{Arten des Maschinellen Lernens}
Hier unterteilt man die Lernmethodiken in drei hierarchisch übergeordnete
Klassen, sogenannte Problemklassen. 
\par Zum einen die Klasse des \textbf{"`Supervised Learnings"'}, die darauf
abzielt, dass bei allen Beispielen Kennzeichungen gegeben sind und hieraus ein 
Zusammenhang zwischen den Beschreibungen der Beispiele und ihren Kennzeichnungen
ermittelt werden soll. \par Die zweite Problemklasse stellt das sogenannte
\textbf{"`Unsupervised Learning"'} dar. Dabei sind zu den Beispielen keine Kennzeichungen gegeben. 
\par Die letzte und dritte Klasse dieser Lernverfahren ist die Art des
\textbf{"`Reinforcement Learnings"'}, die Gegenstand und Schwerpunkt des
nächsten Kapitel sein wird.