\section{Neuronale Netze}
	\subsection{Einführung}
		\begin{frame}
			\frametitle{Einführung}

			\begin{itemize}
		      \item große Menge an vernetzten Neuronen
		      \item Gewichte an den Kanten simulieren selektive Weitergabe von echten
		      Neuronen
		      \item mehrere Schichten:
		      	\begin{itemize}
		            \item Eingabeschicht, mit allen Eingangsparamtern verbunden
		            \item verdeckte Schicht(en), mit Ausgängen der Eingabeschicht verbunden
		            \item Ausgabeschicht, mit Ausgängen der verdeckten Schicht verbunden
		          \end{itemize}
		    \end{itemize}
		\end{frame}

	\subsection{Bedeutung für Reinforcement Learning}

		\begin{frame}
	    \frametitle{Bedeutung für Reinforcement Learning}
	    Abbildung der Q-Funktion als Neuronales Netz
		  \begin{center}
			  \pgfimage[width=0.75\textwidth]{../images/rl-neuronales-netz}
		  \end{center}
	    \end{frame}