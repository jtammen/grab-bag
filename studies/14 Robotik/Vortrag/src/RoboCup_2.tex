\section{RoboCup Fortsetzung}
\subsection{Lernen von Teamfähigkeiten}
\begin{frame}
  \frametitle{"`Mindstormers"': Lernen von Teamfähigkeiten}

  \begin{itemize}
    \item Problem: Anzahl der Spieler wirkt sich exponentiell auf Aktionsmenge
    aus 
    \item Kooperation erreichen durch: gemeinsames Reinforcement-Signal
    \item Ablauf der Lernphase: (überwachtes Lernen)
	  \begin{enumerate}
        \item Ermittlung der erfolgreichen Aktionen
        \item Berechnung des Folgezustands (approximativ)
        \item Bewertung des Folgezustands über neuronales Netz
        \item Auswahl der Aktion mit geringstem Wert
	  \end{enumerate}
  \end{itemize}
\end{frame}

\subsection{Middle-Size-Liga}
\begin{frame}
  \frametitle{Middle-Size-Liga: "`Tribots"'}

  \begin{columns}[b]
    \column{.5\textwidth}
	  \begin{block}{Regeln der Middle-Size-Liga}
	      \begin{itemize}
	        \item<2-> bis zu 4 Roboter pro Team
	        \item<2-> max. 50\,cm Durchmesser
	        \item<2-> Spielfeld: 12 x 8\,m
	        \item<2-> Dauer: 2 x 10 Minuten
	      \end{itemize}
	  \end{block}
    \column{.5\textwidth}
      \pgfimage<2->[height=3.40cm]{../images/robocup-middle}
  \end{columns}
  
  Herausforderungen:  
  \begin{itemize}
    \item Selbstlokalisation durch Kamera
    \item kooperatives Spiel
    \item begrenzte Rechenkapazität (onboard-Rechner)
  \end{itemize}
\end{frame}

\begin{frame}
  \frametitle{Verarbeitung von Sensorinformationen}
  
  \begin{itemize}
    \item Sammeln von Daten über Sensoren (Kamera und Odometrie)
    \item Bestimmen der Position über Spielfeldmarkierungen
    \item Bestimmen der Ballposition und -geschwindigkeit
    \item begrenzte Rechenleistung erfordert sehr simple Bildverarbeitung
    (Farberkennung)
    \item Erstellen eines Umweltbildes aus den Sensordaten, darauf basierend
    Entscheidung fällen
  \end{itemize}
\end{frame}

\begin{frame}
  \frametitle{Lernen von Einzelfähigkeiten und Regelungsaufgaben}
  
  \begin{itemize}
    \item Verfahren aus Simulation nicht direkt übertragbar
    \item Ansatz: Training an simuliertem Roboter, dann übertragen auf realen
    \item Allerdings geht optimale Eigenschaft verloren
    \item Lösung: optimierter Q-Learning-Algorithmus, erlaubt Training am realen
    Roboter mit geringem Aufwand
    \item Ansteuerung der Fahrwerksmotoren auch durch RL lernbar
  \end{itemize}
\end{frame}

\begin{frame}
  \frametitle{Fazit Reinforcement Learning}
  
  \begin{itemize}
    \item Koexistenz von handprogrammierten und gelernten Modulen möglich.
    \item Erlernte Fähigkeiten waren den handgefertigten anfangs überlegen.
    \item Zunehmend Ersetzen der erlernten durch handprogrammierte, optimierte
    Module. 
    \item Mögliche Anwendungsgebiete: Regelungstechnik in der Industrie,
    reaktives Scheduling.
  \end{itemize}
\end{frame}

% \begin{frame}
%   \begin{itemize}
%     \item<1-> First point, shown on all slides.
%     \item<2-> Second point, shown on slide 2 and later.
%     \item<2-> Third point, also shown on slide 2 and later.
%     \item<3-> Fourth point, shown on slide 3.
%   \end{itemize}
% \end{frame}
% \begin{frame}
%   \begin{enumerate}
%     \item<3-| alert@3>[0.] A zeroth point, shown at the very end.
%     \item<1-| alert@1> The first and main point.
%     \item<2-| alert@2> The second point.
%   \end{enumerate}
% \end{frame}

% \begin{frame}<1>[label=zooms]
%   \frametitle<1>{A Complicated Picture}
%   \framezoom<1><2>[border](0cm,0cm)(2cm,1.5cm)
%   \framezoom<1><3>[border](1cm,3cm)(2cm,1.5cm)
%   \framezoom<1><4>[border](3cm,2cm)(3cm,2cm)
%   \pgfimage[height=8cm]{complicatedimagefilename}
% \end{frame}
% \againframe<2->[plain]{zooms}