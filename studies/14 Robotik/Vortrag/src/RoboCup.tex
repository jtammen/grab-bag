\section{RoboCup}
\subsection{Einführung}
\begin{frame}
  \frametitle{Das RoboCup-Projekt}
  
  \begin{itemize}
    \item internationales Forschungs- und Bildungsprojekt
    \item Ziel: Förderung der Erforschung von KI und mobilen autonomen Robotern
    \item Veranstaltung von Wettbewerben
  \end{itemize}
  
  \begin{quote}
  By the year 2050, develop a team of fully autonomous humanoid robots that
  can win against the human world soccer champion team.
  \end{quote}
\end{frame}

\begin{frame}
  \frametitle{Der RoboCup}
  Internationale Wettbewerbs- und Konferenzveranstaltung

  \begin{columns}[b]
    \column{.5\textwidth}
    \begin{block}{Ligen}
      \begin{itemize}
        \item Simulation (2D und 3D)
        \item Small-Size
        \item Middle-Size
        \item Four-Legged
        \item Humanoid
      \end{itemize}
    \end{block}
    
    \column{.5\textwidth}
    \pgfimage<5>[height=3.90cm]{../images/robocup-small}
    \pgfimage<6>[height=3.90cm]{../images/robocup-middle}
    \pgfimage<7>[height=3.90cm]{../images/robocup-4legged}
    \pgfimage<8>[height=3.90cm]{../images/robocup-humanoid}
  \end{columns}
  
  \begin{itemize}
    \item RoboCupRescue: Rescue Robots und Simulation
    \item RoboCupJunior: Soccer, Rescue, Dance
  \end{itemize}
\end{frame}

\subsection{Die "`Mindstormers Tribots"'}
\begin{frame}
  \frametitle{Simulationsliga: "`Mindstormers"'}
  
  \begin{block}{Funktionsweise der Simulation}
      Client-Server-Architektur\\
      Server simuliert Weltmodell und liefert Sensordaten an Clients\\
      11 Clients (eigener Thread) pro Team\\
      pro Zyklus (6000 à 100\,ms) Ausführen eines Kommandos
  \end{block}

  Herausforderungen:  
  \begin{itemize}
    \item verrauschte Sensorinformationen, nur indirekte Kommunikation möglich
    \item kooperatives Spiel
    \item begrenzte Kondition
  \end{itemize}
  
%   \movie[label=cells,width=4cm,height=3cm,poster,showcontrols,duration=5s]{}{../images/2vs2.mpeg}
%   \hyperlinkmovie[start=5s,duration=7s]{cells}{%
%     \beamerbutton{Show the middle stage}%
%   }
%   \hyperlinkmovie[start=12s,duration=5s]{cells}{% 
%   \beamerbutton{Show the late stage}%
%   }
\end{frame}

\begin{frame}
  \frametitle{Visualisierung der Simulation}
  \begin{center}
	  \pgfimage[width=0.80\textwidth]{../images/rcssmonitor_classic}
  \end{center}
\end{frame}

\begin{frame}
  \frametitle{Lernen von Einzelfähigkeiten}
  
  \begin{itemize}
    \item Beispiele: Abfangen eines Balls, ballhalten, dribbeln.
    \item zusammengesetzt aus Elementaraktionen wie \textsl{kick()} oder
    \textsl{dash()}. 
    \item Ablauf des Trainings:
	\begin{enumerate}
      \item zufälliger Startzustand
      \item Auswahl einer Elementaraktion mit niedrigen Kosten
      \item nach Kostenvergabe: Aktualisieren der Wertefunktion
    \end{enumerate}
  \end{itemize}
\end{frame}