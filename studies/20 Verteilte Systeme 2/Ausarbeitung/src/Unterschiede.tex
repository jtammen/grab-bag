
\section{Grundlegende Unterschiede}
\begin{enumerate}
  \item Pastry vergibt 128 bit Schlüssel, während Tapestry 160 bit Schlüssel
  verwendet. Tapestry ist daher für prinzipiell für mehr teilnehmende Peers
  gerüstet.
  \item Tapestry unterscheidet zwischen Knoten (Peers) und Objekten, dies
  impliziert, dass in Tapestry zwischen Routing zu Knoten und Routing zu
  Objekten unterschieden werden muss. Desweitern müssen Knoten Objekte
  publizieren können, die sie bereitstellen. Pastry macht hier keine
  Unterscheidung.
  \item In Tapestry ist eine Maßnahme für den gewollten Austritt eines Peers
  vorgesehen, während Pastry dies immer als gewollten Fehlerfall behandelt.
\end{enumerate}

\section{Pastry vs. Tapestry}
Beide Verfahren beruhen auf dem selben Routing-Prinzip von Plaxton, Rajamaran
und Richa \cite{PlaxtonRR97}, trotzdem zeichnen sie sich durch unterschiedliche Eigenschaften aus.
\newline \newline
Tapestry ist nicht vollständig selbstorganisierend, achtet aber stark auf die
Konsistenz der Routing-Tabelle. Außerdem basiert Tapestry sehr stark auf
analytischen Methoden. Desweiteren ist die Performanz von Tapestry nachweisbar
gut. 
\newline \newline
Pastry hingegen verzichtet weitestgehendst auf analytische Methodik und
setzt dafür Heuristiken ein. Experimentelle Methoden ersetzen also die
analytischen Betrachtungen. Praktisch ist Pastry dadurch gut umsetzbar, auch
wenn der Algorithmus nur ungenau beschrieben ist - z. B. wird nicht beschrieben
wie das Neighbourhood Set (siehe \ref{struktur}) abgebildet ist. Beachtenswert
ist allerdings dass Pastry sehr robust ist, da die Leaf Sets bereits die
Zustellung jedes Datenpaketes zusichern (vgl. \cite{Schindelhauer2004}).
 

