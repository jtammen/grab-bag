Pastry und Tapestry sind Vertreter der neuen Generation von 
Peer-to-Peer-Systemen. Der Einsatz von modernen Routing-Algorithmen und 
topologischen Optimierungen garantiert für beide Systeme relativ kurze 
Laufzeiten. Tapestry hebt sich durch sein ausgeklügeltes Objektmanagement und 
durch die Berücksichtigung des Lokalitätsprinzips hervor, Pastry hat den 
Vorteil weniger komplex zu sein und lässt im Allgemeinen mehr Freiheiten beim 
Aufbau eines solchen Systems. Letztendlich jedoch gibt es kein perfektes 
Peer-to-Peer-System, auch die hier nicht behandelten Systeme \textsl{CAN} und 
\textsl{CHORD} sind nur ein "`Trade-Off"' zwischen Kosten und Leistung.
