Eine ganze Reihe von Peer-to-Peer Applikationen setzen auf der Technik von
Pastry bzw. Tapestry als zugrundeliegende Peer-to-Peer-Netzwerkinfrastruktur auf. 

\section{Tapestry in der Praxis}
Ein Beispiel für die Anwendung von Tapestry in der Praxis wäre
\textsl{OceanStore}. OceanStore ist ein global verteilter, hoch-verfügbarer
Datenspeicher. Hierbei werden die Daten des Netzwerks redundant auf den
angeschlossenen Systemen gehalten, die Zugriffe auf die jeweilig angeforderten
Daten erfolgt jeweils auf nahe Kopien. Ein weiteres, auf der Tapestry-Technik
basierendes Peer-to-Peer-System wäre \textsl{Bayeux.} Bayeux ist ein
Multicast-System, das auf die gleichzeitige Übermittlung von Multimedialen
Inhalten (Beispielsweise Audio- und Videostreams) an eine große Zahl von
Empfängern ausgerichtet ist. Ein drittes Beispiel wäre \textsl{SpamWatch.}
SpamWatch ist ein globales Spam-Filterungssystem für E-Mails, dass für alle
beteiligten Systeme Informationen über Spam-Mails publiziert.

\section{Pastry in der Praxis}
\textsl{PAST} ist ein auf Pastry basierendes verteiltes Speichersystem. Das 
Ziel von PAST ist es, Dateien in einem verteilten System so zu verteilen, dass 
man diese auch dann finden kann, wenn einzelne Knoten im Netz ausgefallen sind. 
Ein weiteres konkretes Anwendungsgebiet von Pastry ist \textsl{Scribe.} Scribe 
ist ein Kommunikationssystem, dass Nachrichten verteilt in einem 
Pastry-Netzwerk publiziert.
