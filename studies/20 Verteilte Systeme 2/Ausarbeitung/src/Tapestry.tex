\textbf{Tapestry} \cite{TapestryTR}, \cite{tapestry_jsac} ist ein Peer-to-Peer 
Overlaynetzwerk, das usprünglich im Jahre 2000 von Ben Y. Zhao, John 
Kubiatowicz und Anthony D. Joseph an der University of California, Berkeley, 
entwickelt wurde. Der grundlegende Mechanismus von Tapestry basiert in 
wesentlichen Teilen auf Arbeiten von Plaxton, Rajaraman und Richa 
\cite{PlaxtonRR97} für statische Netzwerke. Für die praktische Umsetzung von 
Tapestry und anderen verwandten Algorithmen existieren mittlerweile mehrere 
frei verfügbare Implementierungen. Speziell für Tapestry existiert eine 
Java-Implementierung, die jedoch seit Juli 2004 nicht mehr weiterentwickelt wird
und langsam von einem neuen, aktuell existierenden Projekt namens 
'Chimera'\footnote{http://current.cs.ucsb.edu/projects/chimera/} abgelöst 
werden soll. 'Chimera' stellt grundlegende Funktionalität der 'prefix-routing' 
Protokolle Tapestry und Pastry bereit und ist in der Programmiersprache C 
geschrieben.

\section{Grundlagen}
Tapestry unterscheidet zwischen \textsl{Peers} (Knoten) und \textsl{Objekten}.
Objekte können z.B. Dateien sein, die auf einem Knoten zur Verfügung gestellt 
und von anderen Peers abgerufen werden sollen.

Die in einem Tapestry-Overlaynetzwerk eingebundenen Knoten und Objekte erhalten 
als eindeutige Identifizierung eine 160-Bit lange ID. Diese kann zufällig oder 
alternativ unter Verwendung eines Hash-Verfahrens (z.B. SHA-1) zugewiesen 
werden. Bei Verwendung einer Hash-Funktion kann davon ausgegangen werden, dass 
die Werte annährend gleichverteilt sind. Somit ist sichergestellt, dass aus der 
Nachbarschaft zweier Identifikatoren keine Rückschlüsse auf z.B. deren 
geographische Lage gezogen werden können.

Auf einem physikalischen Rechner können mehrere Tapestry-Knoten existieren - 
auf diese Weise lassen sich gleichzeitig unterschiedliche P2P-Applikationen 
über das selbe Overlaynetzwerk betreiben.

\section{Struktur}
Tapestry-Knoten sind in einer Komponentenarchitektur organisiert. Um 
Nachrichten innerhalb des Netzwerks verschicken und empfangen zu können, muss 
in jedem Knoten die Routingtabelle (bestehend aus den Nachbarschaftsmengen) 
verwaltet werden. Weiterhin unterhält jeder Knoten eine Liste sog. 
\textsl{Backpointer}. Dabei handelt es sich um Verweise auf diejenigen Knoten, 
die diesen Knoten als Nachbarn besitzen.

Schließlich besitzt jeder Knoten noch eine Menge von Objektverweisen in der 
Tupelform $\langle \text{Objekt-ID} \, O_G, \text{Knoten-ID} \, N_{id} 
\rangle$, mit deren Hilfe später die Suche nach Objekten deutlich beschleunigt 
werden kann.

\subsection{Nachbarschaftsmengen}
Die Nachbarn eines Knotens $N$ sind in mehrere Mengen unterteilt. Jede dieser 
Nachbarschaftsmengen enthält die Nachbarn einer Ebene $l$, d.h. diejenigen 
Knoten, die einen bestimmten Präfix $x$ mit dem Identifikator $N_{id}$ des 
Knotens $N$ gemeinsam haben. Abbildung \ref{fig:tapestry-neighbor-map} zeigt 
beispielhaft die Nachbarschaftsmengen eines Knotens mit der ID \texttt{4227} 
für die Level 1 bis 4.

\begin{figure}
  \centering
  \def\JPicScale{0.42}
  %%Created by jPicEdt 1.4.1: mixed JPIC-XML/LaTeX format
%%Thu May 31 11:57:59 CEST 2007
%%Begin JPIC-XML
%<?xml version="1.0" standalone="yes"?>
%<jpic x-min="25" x-max="340" y-min="18.38" y-max="305" auto-bounding="true">
%<g>
%<ellipse shadow= "true"
%	 closure= "open"
%	 p3= "(185,150)"
%	 fill-color= "#999999"
%	 shadow-color= "#666666"
%	 p1= "(145,170)"
%	 angle-end= "0"
%	 fill-style= "solid"
%	 stroke-width= "0.5"
%	 p2= "(185,170)"
%	 angle-start= "0"
%	 />
%<text text-vert-align= "center-v"
%	 anchor-point= "(165,160)"
%	 text-hor-align= "center-h"
%	 text-frame= "noframe"
%	 fill-style= "none"
%	 >
%4227
%</text>
%</g>
%<g>
%<ellipse shadow= "true"
%	 closure= "open"
%	 p3= "(245,285)"
%	 fill-color= "#cccccc"
%	 shadow-color= "#666666"
%	 p1= "(205,305)"
%	 angle-end= "0"
%	 fill-style= "solid"
%	 p2= "(245,305)"
%	 angle-start= "0"
%	 />
%<text text-vert-align= "center-v"
%	 anchor-point= "(225,295)"
%	 text-hor-align= "center-h"
%	 text-frame= "noframe"
%	 fill-style= "none"
%	 >
%27AB
%</text>
%</g>
%<g>
%<ellipse shadow= "true"
%	 closure= "open"
%	 p3= "(95,80)"
%	 fill-color= "#cccccc"
%	 shadow-color= "#666666"
%	 p1= "(55,100)"
%	 angle-end= "0"
%	 fill-style= "solid"
%	 p2= "(95,100)"
%	 angle-start= "0"
%	 />
%<text text-vert-align= "center-v"
%	 anchor-point= "(75,90)"
%	 text-hor-align= "center-h"
%	 text-frame= "noframe"
%	 fill-style= "none"
%	 >
%44AF
%</text>
%</g>
%<g>
%<ellipse shadow= "true"
%	 closure= "open"
%	 p3= "(245,200)"
%	 fill-color= "#cccccc"
%	 shadow-color= "#666666"
%	 p1= "(205,220)"
%	 angle-end= "0"
%	 fill-style= "solid"
%	 p2= "(245,220)"
%	 angle-start= "0"
%	 />
%<text text-vert-align= "center-v"
%	 anchor-point= "(225,210)"
%	 text-hor-align= "center-h"
%	 text-frame= "noframe"
%	 fill-style= "none"
%	 >
%42A2
%</text>
%</g>
%<g>
%<ellipse shadow= "true"
%	 closure= "open"
%	 p3= "(300,50)"
%	 fill-color= "#cccccc"
%	 shadow-color= "#666666"
%	 p1= "(260,70)"
%	 angle-end= "0"
%	 fill-style= "solid"
%	 p2= "(300,70)"
%	 angle-start= "0"
%	 />
%<text text-vert-align= "center-v"
%	 anchor-point= "(280,60)"
%	 text-hor-align= "center-h"
%	 text-frame= "noframe"
%	 fill-style= "none"
%	 >
%51E5
%</text>
%</g>
%<g>
%<ellipse shadow= "true"
%	 closure= "open"
%	 p3= "(230,45)"
%	 fill-color= "#cccccc"
%	 shadow-color= "#666666"
%	 p1= "(190,65)"
%	 angle-end= "0"
%	 fill-style= "solid"
%	 p2= "(230,65)"
%	 angle-start= "0"
%	 />
%<text text-vert-align= "center-v"
%	 anchor-point= "(210,55)"
%	 text-hor-align= "center-h"
%	 text-frame= "noframe"
%	 fill-style= "none"
%	 >
%43C9
%</text>
%</g>
%<g>
%<ellipse shadow= "true"
%	 closure= "open"
%	 p3= "(65,250)"
%	 fill-color= "#cccccc"
%	 shadow-color= "#666666"
%	 p1= "(25,270)"
%	 angle-end= "0"
%	 fill-style= "solid"
%	 p2= "(65,270)"
%	 angle-start= "0"
%	 />
%<text text-vert-align= "center-v"
%	 anchor-point= "(45,260)"
%	 text-hor-align= "center-h"
%	 text-frame= "noframe"
%	 fill-style= "none"
%	 >
%6F43
%</text>
%</g>
%<g>
%<ellipse shadow= "true"
%	 closure= "open"
%	 p3= "(175,190)"
%	 fill-color= "#cccccc"
%	 shadow-color= "#666666"
%	 p1= "(135,210)"
%	 angle-end= "0"
%	 fill-style= "solid"
%	 p2= "(175,210)"
%	 angle-start= "0"
%	 />
%<text text-vert-align= "center-v"
%	 anchor-point= "(155,200)"
%	 text-hor-align= "center-h"
%	 text-frame= "noframe"
%	 fill-style= "none"
%	 >
%4228
%</text>
%</g>
%<g>
%<ellipse shadow= "true"
%	 closure= "open"
%	 p3= "(340,180)"
%	 fill-color= "#cccccc"
%	 shadow-color= "#666666"
%	 p1= "(300,200)"
%	 angle-end= "0"
%	 fill-style= "solid"
%	 p2= "(340,200)"
%	 angle-start= "0"
%	 />
%<text text-vert-align= "center-v"
%	 anchor-point= "(320,190)"
%	 text-hor-align= "center-h"
%	 text-frame= "noframe"
%	 fill-style= "none"
%	 >
%1D76
%</text>
%</g>
%<multicurve left-arrow= "head"
%	 fill-style= "none"
%	 points= "(155,190);(155,190);(158.95,169.65);(158.95,169.65)"
%	 />
%<multicurve left-arrow= "head"
%	 fill-style= "none"
%	 points= "(208.69,204.17);(208.69,204.17);(178.34,167.42);(178.34,167.42)"
%	 />
%<circle closure= "plain"
%	 p3= "(186.48,149.33)"
%	 stroke-dasharray= "1;1"
%	 p1= "(153.89,181.26)"
%	 stroke-color= "#333333"
%	 fill-style= "none"
%	 p2= "(188.98,159.51)"
%	 stroke-style= "dashed"
%	 />
%<text text-vert-align= "center-v"
%	 anchor-point= "(158.87,144)"
%	 text-hor-align= "center-h"
%	 text-frame= "noframe"
%	 fill-style= "none"
%	 >
%$l=5$
%</text>
%<circle closure= "plain"
%	 p3= "(224.44,161.98)"
%	 stroke-dasharray= "1;1"
%	 p1= "(179.73,217.62)"
%	 stroke-color= "#333333"
%	 fill-style= "none"
%	 p2= "(214.73,192.62)"
%	 stroke-style= "dashed"
%	 />
%<text pstcustom= "linecolor=red"
%	 text-vert-align= "center-v"
%	 anchor-point= "(148.69,119.79)"
%	 text-hor-align= "center-h"
%	 text-frame= "noframe"
%	 fill-style= "none"
%	 >
%$l=4$
%</text>
%<multicurve left-arrow= "head"
%	 fill-style= "none"
%	 points= "(87.73,97.86);(87.73,97.86);(146.67,155.84);(146.67,155.84)"
%	 />
%<circle closure= "plain"
%	 p3= "(260.63,173.02)"
%	 stroke-dasharray= "1;1"
%	 p1= "(215.4,242.31)"
%	 stroke-color= "#333333"
%	 fill-style= "none"
%	 p2= "(253.48,198.55)"
%	 stroke-style= "dashed"
%	 />
%<text pstcustom= "linecolor=red"
%	 text-vert-align= "center-v"
%	 anchor-point= "(133.25,87.51)"
%	 text-hor-align= "center-h"
%	 text-frame= "noframe"
%	 fill-style= "none"
%	 >
%$l=3$
%</text>
%<circle closure= "plain"
%	 p3= "(296.42,183.04)"
%	 stroke-dasharray= "1;1"
%	 p1= "(246.7,265.48)"
%	 stroke-color= "#333333"
%	 fill-style= "none"
%	 p2= "(289.27,208.57)"
%	 stroke-style= "dashed"
%	 />
%<text pstcustom= "linecolor=red"
%	 text-vert-align= "center-v"
%	 anchor-point= "(114.66,54.17)"
%	 text-hor-align= "center-h"
%	 text-frame= "noframe"
%	 fill-style= "none"
%	 >
%$l=2$
%</text>
%<text pstcustom= "linecolor=red"
%	 text-vert-align= "center-v"
%	 anchor-point= "(94.66,18.38)"
%	 text-hor-align= "center-h"
%	 text-frame= "noframe"
%	 fill-style= "none"
%	 >
%$l=1$
%</text>
%<multicurve left-arrow= "head"
%	 fill-style= "none"
%	 points= "(200.97,64);(200.97,64);(165,150);(165,150)"
%	 />
%<multicurve left-arrow= "head"
%	 fill-style= "none"
%	 points= "(263.16,65.4);(263.16,65.4);(174.3,151.28);(174.3,151.28)"
%	 />
%<multicurve left-arrow= "head"
%	 fill-style= "none"
%	 points= "(300,190);(300,190);(185,160);(185,160)"
%	 />
%<multicurve left-arrow= "head"
%	 fill-style= "none"
%	 points= "(56.94,251.89);(56.94,251.89);(148.16,165.66);(148.16,165.66)"
%	 />
%<multicurve left-arrow= "head"
%	 fill-style= "none"
%	 points= "(213.16,286.8);(213.16,286.8);(170.88,169.7);(170.88,169.7)"
%	 />
%</jpic>
%%End JPIC-XML
%PSTricks content-type (pstricks.sty package needed)
%Add \usepackage{pstricks} in the preamble of your LaTeX file
%You can rescale the whole picture (to 80% for instance) by using the command \def\JPicScale{0.8}
\ifx\JPicScale\undefined\def\JPicScale{1}\fi
\psset{unit=\JPicScale mm}
\psset{linewidth=0.3,dotsep=1,hatchwidth=0.3,hatchsep=1.5,shadowsize=1,dimen=middle}
\psset{dotsize=0.7 2.5,dotscale=1 1,fillcolor=black}
\psset{arrowsize=1 2,arrowlength=1,arrowinset=0.25,tbarsize=0.7 5,bracketlength=0.15,rbracketlength=0.15}
\begin{pspicture}(0,0)(340,305)
\newrgbcolor{userFillColour}{0.6 0.6 0.6}
\newrgbcolor{userShadowColour}{0.4 0.4 0.4}
\rput{0}(165,160){\psellipse[linewidth=0.5,fillcolor=userFillColour,fillstyle=solid,shadow=true,shadowcolor=userShadowColour](0,0)(20,-10)}
\rput(165,160){4227}
\newrgbcolor{userFillColour}{0.8 0.8 0.8}
\newrgbcolor{userShadowColour}{0.4 0.4 0.4}
\rput{0}(225,295){\psellipse[fillcolor=userFillColour,fillstyle=solid,shadow=true,shadowcolor=userShadowColour](0,0)(20,-10)}
\rput(225,295){27AB}
\newrgbcolor{userFillColour}{0.8 0.8 0.8}
\newrgbcolor{userShadowColour}{0.4 0.4 0.4}
\rput{0}(75,90){\psellipse[fillcolor=userFillColour,fillstyle=solid,shadow=true,shadowcolor=userShadowColour](0,0)(20,-10)}
\rput(75,90){44AF}
\newrgbcolor{userFillColour}{0.8 0.8 0.8}
\newrgbcolor{userShadowColour}{0.4 0.4 0.4}
\rput{0}(225,210){\psellipse[fillcolor=userFillColour,fillstyle=solid,shadow=true,shadowcolor=userShadowColour](0,0)(20,-10)}
\rput(225,210){42A2}
\newrgbcolor{userFillColour}{0.8 0.8 0.8}
\newrgbcolor{userShadowColour}{0.4 0.4 0.4}
\rput{0}(280,60){\psellipse[fillcolor=userFillColour,fillstyle=solid,shadow=true,shadowcolor=userShadowColour](0,0)(20,-10)}
\rput(280,60){51E5}
\newrgbcolor{userFillColour}{0.8 0.8 0.8}
\newrgbcolor{userShadowColour}{0.4 0.4 0.4}
\rput{0}(210,55){\psellipse[fillcolor=userFillColour,fillstyle=solid,shadow=true,shadowcolor=userShadowColour](0,0)(20,-10)}
\rput(210,55){43C9}
\newrgbcolor{userFillColour}{0.8 0.8 0.8}
\newrgbcolor{userShadowColour}{0.4 0.4 0.4}
\rput{0}(45,260){\psellipse[fillcolor=userFillColour,fillstyle=solid,shadow=true,shadowcolor=userShadowColour](0,0)(20,-10)}
\rput(45,260){6F43}
\newrgbcolor{userFillColour}{0.8 0.8 0.8}
\newrgbcolor{userShadowColour}{0.4 0.4 0.4}
\rput{0}(155,200){\psellipse[fillcolor=userFillColour,fillstyle=solid,shadow=true,shadowcolor=userShadowColour](0,0)(20,-10)}
\rput(155,200){4228}
\newrgbcolor{userFillColour}{0.8 0.8 0.8}
\newrgbcolor{userShadowColour}{0.4 0.4 0.4}
\rput{0}(320,190){\psellipse[fillcolor=userFillColour,fillstyle=solid,shadow=true,shadowcolor=userShadowColour](0,0)(20,-10)}
\rput(320,190){1D76}
\psline{<-}(155,190)(158.95,169.65)
\psline{<-}(208.69,204.17)(178.34,167.42)
\newrgbcolor{userLineColour}{0.2 0.2 0.2}
\rput{0}(165,160){\psellipse[linecolor=userLineColour,linestyle=dashed,dash=1 1](0,0)(23.99,23.99)}
\rput(158.87,144){{\tiny $l=5$}}
\newrgbcolor{userLineColour}{0.2 0.2 0.2}
\rput{0}(165,160){\psellipse[linecolor=userLineColour,linestyle=dashed,dash=1 1](0,0)(59.47,59.47)}
\rput(148.69,119.79){{\tiny $l=4$}}
\psline{<-}(87.73,97.86)(146.67,155.84)
\newrgbcolor{userLineColour}{0.2 0.2 0.2}
\rput{0}(165,160){\psellipse[linecolor=userLineColour,linestyle=dashed,dash=1 1](0,0)(96.51,96.51)}
\rput(133.25,87.51){{\tiny $l=3$}}
\newrgbcolor{userLineColour}{0.2 0.2 0.2}
\rput{0}(165.02,160.01){\psellipse[linecolor=userLineColour,linestyle=dashed,dash=1 1](0,0)(133.4,133.4)}
\rput(114.66,54.17){{\tiny $l=2$}}
\rput(94.66,18.38){{\tiny $l=1$}}
\psline{<-}(200.97,64)(165,150)
\psline{<-}(263.16,65.4)(174.3,151.28)
\psline{<-}(300,190)(185,160)
\psline{<-}(56.94,251.89)(148.16,165.66)
\psline{<-}(213.16,286.8)(170.88,169.7)
\end{pspicture}

  \caption[Tapestry: Nachbarschaftsmengen]{Nachbarschaftsmengen des
  Tapestry-Knotens \texttt{4227}. Vgl. \cite[Fig. 1]{TapestryTR}}
  \label{fig:tapestry-neighbor-map}
\end{figure}

% Die Nachbarschaftsmengen besitzen nach \cite{Pastry-Middleware} die folgenden
% zwei Eigenschaften.
% 
% \begin{description}
%   \item[Konsistenz.] Sollte die Nachbarschaftsmenge eines bestimmten Levels $l$
%   leer sein, so ist gesichert, dass im gesamten Netz kein Knoten existiert, der 
%   \item[Lokalität.]
% \end{description}

\section{Routing}
Da Tapestry zwischen Knoten und Objekten unterscheidet, müssen auch beim 
Routing Unterschiede getroffen werden. Allgemein basiert das Routing in 
Tapestry auf dem bereits von Pastry bekannten Präfix-Routing.

\subsection{Routing to nodes}
Soll eine Nachricht von einem Knoten $N$ zu einem Knoten $X$ gesendet werden, 
so wird die Nachricht im ersten Schritt an einen Peer gesendet, dessen ID in 
einer einzigen Stelle mit $X_{id}$ übereinstimmt. Dieser Peer sendet die 
Nachricht nun weiter an einen Knoten, dessen ID \textsl{zwei} Stellen mit 
$X_{id}$ gemeinsam hat. Dieser Vorgang setzt sich solange fort, bis die 
Nachricht den Zielknoten $X$ erreicht. Aufgrund der baumartigen 
Routing-Struktur benötigt dieses Verfahren einen logarithmischen Aufwand.

Treten jedoch "`Löcher"' in der Routingtabelle auf, d.h. es ist kein Knoten mit
passendem ID-Präfix bekannt, so könnte es zu Problemen kommen. Abhilfe schafft
das sog. Surrogate-Routing, welches in Abschnitt \ref{subsec:surrogate-routing}
näher beschrieben wird. Abbildung \ref{fig:tapestry-route-node} zeigt
beispielhaft das Routing einer Nachricht von Knoten \texttt{5230} an Knoten
\texttt{42AD}. 

\begin{figure}
  \centering
  \def\JPicScale{0.55}
  \input{../images/tapestry-route-node}
  \caption[Tapestry: Routing zu Knoten]{Senden einer Nachricht vom Knoten
  \texttt{5230} an den Knoten 
  \texttt{42AD}. Vgl. \cite[Fig. 2]{TapestryTR}}
  \label{fig:tapestry-route-node}
\end{figure}

Wie man sieht, läuft die Nachricht auf ihrem Weg zum Ziel stets zu einem 
Knoten, dessen ID in einer höheren Anzahl an Stellen mit der ID des Ziels 
übereinstimmt.

\subsection{Routing to objects}
Nachrichten an Objekte werden analog zu Nachrichten an Knoten mittels des 
Präfix-Routings geroutet und an den sog. Root-Knoten $O_R$ des Objekts $O$ 
gesendet. Der Root-Knoten eines Objekts ist ein vom bereitstellenden Knoten 
verschiedener Knoten, der für das Objekt verantwortlich ist - das bedeutet, 
dass er Links zu denjenigen Knoten speichert, die das Objekt bereithalten. 
Dabei können für das selbe Objekt durchaus mehrere Root-Knoten existieren. In 
Tapestry gibt es dazu eine Funktion $MapRoots()$, welche die ID $O_{G}$ eines 
Objekts $O$ auf die Menge der Root-Knoten, das sog. Root-Set, abbildet. Um zu 
garantieren, dass diese Abbildungsfunktion in jeder Situation ein eindeutiges 
und identisches Root-Set zurückliefert, kann beispielsweise eine Hash-Funktion 
eingesetzt werden \cite{HildrumKRZ02}.

Wie bereits erwähnt, speichert jeder Tapestry-Knoten jedoch noch lokale 
Objektverweise. Das bedeutet, dass das Routing der Nachricht u.U. nicht 
komplett Tapestry nimmt an, dass, je dichter sich der Startknoten zum Objekt 
befindet, desto schneller die Anfrage auf einen Knoten mit Verweis auf dieses 
Objekt trifft, sprich das Objekt schneller gefunden wird.

\begin{figure}
  \centering
  \def\JPicScale{0.59}
  \input{../images/tapestry-route-object}
  \caption[Tapestry: Routing zu Objekt]{Senden einer Nachricht an das Objekt
  \texttt{4378} von verschiedenen Knoten aus. Vgl. \cite[Fig. 5]{TapestryTR}}
  \label{fig:tapestry-route-object}
\end{figure}

Das Beispiel (Abb. \ref{fig:tapestry-route-object}) zeigt, wie drei Knoten
(\texttt{4664}, \texttt{4B4F} und \texttt{57EC}) jeweils auf das Objekt
\texttt{4378} zugreifen möchten. Dabei senden sie ihre Anfrage Richtung $O_R$,
(hier \texttt{4377}) - sobald sie auf einen Knoten trifft, der einen lokalen
Objektverweis gespeichert hat, wird die Nachricht direkt an den bereitstellenden
Knoten weitergeschickt.

\subsection{Publizieren von Objekten}
Um ein solches Objekt $O$ im Netzwerk bekannt zu machen, sendet der 
bereitstellende Knoten $S$ (genannt \textsl{Storage Server}) perdiodische 
\textsl{publish}-Nachrichten in Richtung des Root-Knotens $O_R$ von $O$. Jeder 
Knoten, der diese Nachricht auf ihrem Weg empfängt bzw. weiterleitet, speichert 
nun einen Objektverweis $\langle O_G, S \rangle$. Auf diese Weise muss später 
bei Anfragen an dieses Objekt u.U. nicht der komplette Weg zum Root-Knoten 
durchlaufen werden. Tapestry unterstützt auch Objekt-Kopien, die auf 
verschiedenen Knoten liegen können. Dabei publiziert jeder Knoten für sich 
seine Kopie in das Netzwerk; andere Knoten haben dann u.U. mehrere 
Objektverweise auf das selbe Objekt, die aber auf einen anderen Knoten zeigen. 
Auf diese Weise wird die Redundanz und Zuverlässigkeit des Systems erhöht.

\begin{figure}[hp]
  \centering
  \def\JPicScale{0.59}
  %%Created by jPicEdt 1.4.1: mixed JPIC-XML/LaTeX format
%%Sun Jun 10 12:45:52 CEST 2007
%%Begin JPIC-XML
%<?xml version="1.0" standalone="yes"?>
%<jpic x-min="1.05" x-max="266.12" y-min="3.16" y-max="158.16" auto-bounding="true">
%<g>
%<ellipse fill-style= "solid"
%	 shadow= "true"
%	 p3= "(41.05,48.16)"
%	 fill-color= "#cccccc"
%	 p2= "(41.05,68.16)"
%	 p1= "(1.05,68.16)"
%	 closure= "open"
%	 shadow-color= "#666666"
%	 angle-end= "0"
%	 angle-start= "0"
%	 />
%<text fill-style= "none"
%	 text-vert-align= "center-v"
%	 anchor-point= "(21.05,58.16)"
%	 text-frame= "noframe"
%	 text-hor-align= "center-h"
%	 >
%4228
%</text>
%</g>
%<g>
%<ellipse fill-style= "solid"
%	 shadow= "true"
%	 p3= "(111.05,58.16)"
%	 fill-color= "#cccccc"
%	 p2= "(111.05,78.16)"
%	 p1= "(71.05,78.16)"
%	 closure= "open"
%	 shadow-color= "#666666"
%	 angle-end= "0"
%	 angle-start= "0"
%	 />
%<text fill-style= "none"
%	 text-vert-align= "center-v"
%	 anchor-point= "(91.05,68.16)"
%	 text-frame= "noframe"
%	 text-hor-align= "center-h"
%	 >
%4664
%</text>
%</g>
%<g>
%<ellipse fill-style= "solid"
%	 shadow= "true"
%	 p3= "(71.05,88.16)"
%	 fill-color= "#cccccc"
%	 p2= "(71.05,108.16)"
%	 p1= "(31.05,108.16)"
%	 closure= "open"
%	 shadow-color= "#666666"
%	 angle-end= "0"
%	 angle-start= "0"
%	 />
%<text fill-style= "none"
%	 text-vert-align= "center-v"
%	 anchor-point= "(51.05,98.16)"
%	 text-frame= "noframe"
%	 text-hor-align= "center-h"
%	 >
%43FE
%</text>
%</g>
%<multicurve fill-style= "none"
%	 left-arrow= "head"
%	 points= "(40.39,89.74);(40.39,89.74);(24.47,68.03);(24.47,68.03)"
%	 />
%<multicurve fill-style= "none"
%	 left-arrow= "head"
%	 points= "(64.74,90.86);(64.74,90.86);(77.3,75.39);(77.3,75.39)"
%	 />
%<g>
%<multicurve fill-style= "solid"
%	 shadow= "true"
%	 fill-color= "#ffcccc"
%	 shadow-color= "#000000"
%	 points= "(33.16,33.16);(33.16,33.16);(33.16,3.16);(33.16,3.16);(33.16,3.16);(8.16,3.16);
%	(8.16,3.16);(8.16,3.16);(8.16,28.16);(8.16,28.16);(8.16,28.16);(13.16,33.16);
%	(13.16,33.16);(13.16,33.16);(28.16,33.16);(28.16,33.16);(28.16,33.16);(33.16,33.16)"
%	 />
%<text fill-style= "none"
%	 text-vert-align= "center-v"
%	 anchor-point= "(20.79,23.16)"
%	 text-frame= "noframe"
%	 text-hor-align= "center-h"
%	 >
%Datei X
%</text>
%<text fill-style= "none"
%	 text-vert-align= "center-v"
%	 anchor-point= "(20.79,12.63)"
%	 text-frame= "noframe"
%	 text-hor-align= "center-h"
%	 >
%4378
%</text>
%<multicurve fill-style= "none"
%	 points= "(8.16,28.16);(8.16,28.16);(13.16,28.16);(13.16,28.16);(13.16,28.16);(13.16,33.16);
%	(13.16,33.16)"
%	 />
%</g>
%<multicurve fill-style= "none"
%	 stroke-style= "dashed"
%	 stroke-dasharray= "1;1"
%	 points= "(21.05,48.16);(21.05,48.16);(21.05,33.16);(21.05,33.16)"
%	 />
%<g>
%<ellipse fill-style= "solid"
%	 shadow= "true"
%	 p3= "(171.05,113.16)"
%	 fill-color= "#cccccc"
%	 p2= "(171.05,133.16)"
%	 p1= "(131.05,133.16)"
%	 closure= "open"
%	 shadow-color= "#666666"
%	 angle-end= "0"
%	 angle-start= "0"
%	 />
%<text fill-style= "none"
%	 text-vert-align= "center-v"
%	 anchor-point= "(151.05,123.16)"
%	 text-frame= "noframe"
%	 text-hor-align= "center-h"
%	 >
%437A
%</text>
%</g>
%<g>
%<ellipse fill-style= "solid"
%	 shadow= "true"
%	 p3= "(116.05,138.16)"
%	 fill-color= "#cccccc"
%	 p2= "(116.05,158.16)"
%	 p1= "(76.05,158.16)"
%	 closure= "open"
%	 shadow-color= "#666666"
%	 angle-end= "0"
%	 angle-start= "0"
%	 />
%<text fill-style= "none"
%	 text-vert-align= "center-v"
%	 anchor-point= "(96.05,148.16)"
%	 text-frame= "noframe"
%	 text-hor-align= "center-h"
%	 >
%4377
%</text>
%</g>
%<g>
%<ellipse fill-style= "solid"
%	 shadow= "true"
%	 p3= "(181.05,78.16)"
%	 fill-color= "#cccccc"
%	 p2= "(181.05,98.16)"
%	 p1= "(141.05,98.16)"
%	 closure= "open"
%	 shadow-color= "#666666"
%	 angle-end= "0"
%	 angle-start= "0"
%	 />
%<text fill-style= "none"
%	 text-vert-align= "center-v"
%	 anchor-point= "(161.05,88.16)"
%	 text-frame= "noframe"
%	 text-hor-align= "center-h"
%	 >
%4361
%</text>
%</g>
%<g>
%<ellipse fill-style= "solid"
%	 shadow= "true"
%	 p3= "(216.05,43.16)"
%	 fill-color= "#cccccc"
%	 p2= "(216.05,63.16)"
%	 p1= "(176.05,63.16)"
%	 closure= "open"
%	 shadow-color= "#666666"
%	 angle-end= "0"
%	 angle-start= "0"
%	 />
%<text fill-style= "none"
%	 text-vert-align= "center-v"
%	 anchor-point= "(196.05,53.16)"
%	 text-frame= "noframe"
%	 text-hor-align= "center-h"
%	 >
%4A6D
%</text>
%</g>
%<g>
%<ellipse fill-style= "solid"
%	 shadow= "true"
%	 p3= "(156.05,33.16)"
%	 fill-color= "#cccccc"
%	 p2= "(156.05,53.16)"
%	 p1= "(116.05,53.16)"
%	 closure= "open"
%	 shadow-color= "#666666"
%	 angle-end= "0"
%	 angle-start= "0"
%	 />
%<text fill-style= "none"
%	 text-vert-align= "center-v"
%	 anchor-point= "(136.05,43.16)"
%	 text-frame= "noframe"
%	 text-hor-align= "center-h"
%	 >
%4B4F
%</text>
%</g>
%<g>
%<ellipse fill-style= "solid"
%	 shadow= "true"
%	 p3= "(121.05,3.16)"
%	 fill-color= "#cccccc"
%	 p2= "(121.05,23.16)"
%	 p1= "(81.05,23.16)"
%	 closure= "open"
%	 shadow-color= "#666666"
%	 angle-end= "0"
%	 angle-start= "0"
%	 />
%<text fill-style= "none"
%	 text-vert-align= "center-v"
%	 anchor-point= "(101.05,13.16)"
%	 text-frame= "noframe"
%	 text-hor-align= "center-h"
%	 >
%E791
%</text>
%</g>
%<g>
%<ellipse fill-style= "solid"
%	 shadow= "true"
%	 p3= "(201.05,8.16)"
%	 fill-color= "#cccccc"
%	 p2= "(201.05,28.16)"
%	 p1= "(161.05,28.16)"
%	 closure= "open"
%	 shadow-color= "#666666"
%	 angle-end= "0"
%	 angle-start= "0"
%	 />
%<text fill-style= "none"
%	 text-vert-align= "center-v"
%	 anchor-point= "(181.05,18.16)"
%	 text-frame= "noframe"
%	 text-hor-align= "center-h"
%	 >
%57EC
%</text>
%</g>
%<g>
%<ellipse fill-style= "solid"
%	 shadow= "true"
%	 p3= "(256.05,3.16)"
%	 fill-color= "#cccccc"
%	 p2= "(256.05,23.16)"
%	 p1= "(216.05,23.16)"
%	 closure= "open"
%	 shadow-color= "#666666"
%	 angle-end= "0"
%	 angle-start= "0"
%	 />
%<text fill-style= "none"
%	 text-vert-align= "center-v"
%	 anchor-point= "(236.05,13.16)"
%	 text-frame= "noframe"
%	 text-hor-align= "center-h"
%	 >
%AA93
%</text>
%</g>
%<multicurve fill-style= "none"
%	 left-arrow= "head"
%	 points= "(131.78,120.33);(131.78,120.33);(70.99,99.01);(70.99,99.01)"
%	 />
%<multicurve fill-style= "none"
%	 left-arrow= "head"
%	 points= "(113.68,143.36);(113.68,143.36);(134.28,128.75);(134.28,128.75)"
%	 />
%<multicurve fill-style= "none"
%	 left-arrow= "head"
%	 points= "(151.05,113.16);(151.05,113.16);(157.89,98.03);(157.89,98.03)"
%	 />
%<multicurve fill-style= "none"
%	 left-arrow= "head"
%	 points= "(156.05,78.16);(156.05,78.16);(136.05,53.16);(136.05,53.16)"
%	 />
%<multicurve fill-style= "none"
%	 left-arrow= "head"
%	 points= "(120.33,36.64);(120.33,36.64);(104.08,23.03);(104.08,23.03)"
%	 />
%<multicurve fill-style= "none"
%	 left-arrow= "head"
%	 points= "(176.18,81.51);(176.18,81.51);(193.09,63.16);(193.09,63.16)"
%	 />
%<multicurve fill-style= "none"
%	 left-arrow= "head"
%	 points= "(188.42,43.88);(188.42,43.88);(183.36,28.22);(183.36,28.22)"
%	 />
%<multicurve fill-style= "none"
%	 left-arrow= "head"
%	 points= "(205,44.14);(205,44.14);(220.26,19.34);(220.26,19.34)"
%	 />
%<g>
%<multicurve fill-style= "solid"
%	 shadow= "true"
%	 fill-color= "#ffcccc"
%	 shadow-color= "#000000"
%	 points= "(266.12,67.76);(266.12,67.76);(266.12,37.76);(266.12,37.76);(266.12,37.76);(241.12,37.76);
%	(241.12,37.76);(241.12,37.76);(241.12,62.76);(241.12,62.76);(241.12,62.76);(246.12,67.76);
%	(246.12,67.76);(246.12,67.76);(261.12,67.76);(261.12,67.76);(261.12,67.76);(266.12,67.76)"
%	 />
%<text fill-style= "none"
%	 text-vert-align= "center-v"
%	 anchor-point= "(253.75,57.76)"
%	 text-frame= "noframe"
%	 text-hor-align= "center-h"
%	 >
%Datei X
%</text>
%<text fill-style= "none"
%	 text-vert-align= "center-v"
%	 anchor-point= "(253.75,47.24)"
%	 text-frame= "noframe"
%	 text-hor-align= "center-h"
%	 >
%4378
%</text>
%<multicurve fill-style= "none"
%	 points= "(241.12,62.76);(241.12,62.76);(246.12,62.76);(246.12,62.76);(246.12,62.76);(246.12,67.76);
%	(246.12,67.76)"
%	 />
%</g>
%<multicurve fill-style= "none"
%	 stroke-style= "dashed"
%	 stroke-dasharray= "1;1"
%	 points= "(253.95,37.57);(253.95,37.57);(247.24,21.38);(247.24,21.38)"
%	 />
%<smoothpolygon fill-style= "none"
%	 stroke-style= "dashed"
%	 stroke-dasharray= "1;1"
%	 left-arrow= "head"
%	 closed= "false"
%	 points= "(15.07,67.83);(15.53,96.58);(30.99,98.62)"
%	 smoothness= "0.7;0.7;0.7"
%	 />
%<smoothpolygon fill-style= "none"
%	 stroke-style= "dashed"
%	 stroke-dasharray= "1;1"
%	 left-arrow= "head"
%	 closed= "false"
%	 points= "(10,66.84);(5.26,132.11);(131.32,124.87)"
%	 smoothness= "0.7;0.7;0.7"
%	 />
%<smoothpolygon fill-style= "none"
%	 stroke-style= "dashed"
%	 stroke-dasharray= "1;1"
%	 left-arrow= "head"
%	 closed= "false"
%	 points= "(4.93,64.14);(4.21,141.05);(76.32,149.01)"
%	 smoothness= "0.7;0.7;0.7"
%	 />
%<smoothpolygon fill-style= "none"
%	 stroke-style= "dashed"
%	 stroke-dasharray= "1;1"
%	 left-arrow= "head"
%	 closed= "false"
%	 points= "(239.67,23.16);(217.89,146.32);(115.86,149.41)"
%	 smoothness= "0.7;0.7;0.7"
%	 />
%<smoothpolygon fill-style= "none"
%	 stroke-style= "dashed"
%	 stroke-dasharray= "1;1"
%	 left-arrow= "head"
%	 closed= "false"
%	 points= "(236.32,23.16);(216.32,100.66);(170.99,122.3)"
%	 smoothness= "0.7;0.7;0.7"
%	 />
%<smoothpolygon fill-style= "none"
%	 stroke-style= "dashed"
%	 stroke-dasharray= "1;1"
%	 left-arrow= "head"
%	 closed= "false"
%	 points= "(232.11,22.89);(214.74,86.58);(180.92,87.89)"
%	 smoothness= "0.7;0.7;0.7"
%	 />
%<smoothpolygon fill-style= "none"
%	 stroke-style= "dashed"
%	 stroke-dasharray= "1;1"
%	 left-arrow= "head"
%	 closed= "false"
%	 points= "(225,21.58);(220,40.53);(212.11,46.84)"
%	 smoothness= "0.7;0.7;0.7"
%	 />
%<g>
%<multicurve fill-style= "none"
%	 stroke-style= "dashed"
%	 stroke-dasharray= "1;1"
%	 points= "(240,125);(240,125);(265,125);(265,125)"
%	 right-arrow= "head"
%	 />
%<text fill-style= "none"
%	 text-vert-align= "center-v"
%	 anchor-point= "(250,120)"
%	 text-frame= "noframe"
%	 text-hor-align= "center-h"
%	 >
%Objektverweis
%</text>
%</g>
%<g>
%<multicurve fill-style= "none"
%	 points= "(240,140);(240,140);(265,140);(265,140)"
%	 right-arrow= "head"
%	 />
%<text fill-style= "none"
%	 text-vert-align= "center-v"
%	 anchor-point= "(250,135)"
%	 text-frame= "noframe"
%	 text-hor-align= "center-h"
%	 >
%Tapestry-Link
%</text>
%</g>
%<multicurve fill-style= "none"
%	 stroke-color= "#ff0000"
%	 points= "(240,155);(240,155);(265,155);(265,155)"
%	 stroke-width= "0.5"
%	 right-arrow= "head"
%	 />
%<text fill-style= "none"
%	 stroke-color= "#ff0000"
%	 text-vert-align= "center-v"
%	 anchor-point= "(250,150)"
%	 stroke-width= "0.5"
%	 text-frame= "noframe"
%	 text-hor-align= "center-h"
%	 >
%Publish-Weg
%</text>
%<multicurve fill-style= "none"
%	 stroke-color= "#ff0000"
%	 left-arrow= "head"
%	 points= "(46.84,93.68);(46.84,93.68);(25.26,63.95);(25.26,63.95)"
%	 stroke-width= "0.5"
%	 />
%<multicurve fill-style= "none"
%	 stroke-color= "#ff0000"
%	 left-arrow= "head"
%	 points= "(136.58,125.26);(136.58,125.26);(61.32,99.21);(61.32,99.21)"
%	 stroke-width= "0.5"
%	 />
%<multicurve fill-style= "none"
%	 stroke-color= "#ff0000"
%	 left-arrow= "head"
%	 points= "(205.53,49.74);(205.53,49.74);(225.53,17.11);(225.53,17.11)"
%	 stroke-width= "0.5"
%	 />
%<multicurve fill-style= "none"
%	 stroke-color= "#ff0000"
%	 left-arrow= "head"
%	 points= "(168.68,84.21);(168.68,84.21);(191.32,59.47);(191.32,59.47)"
%	 stroke-width= "0.5"
%	 />
%<multicurve fill-style= "none"
%	 stroke-color= "#ff0000"
%	 left-arrow= "head"
%	 points= "(152.11,118.16);(152.11,118.16);(162.89,94.74);(162.89,94.74)"
%	 stroke-width= "0.5"
%	 />
%<multicurve fill-style= "none"
%	 stroke-color= "#ff0000"
%	 left-arrow= "head"
%	 points= "(110.79,149.47);(110.79,149.47);(141.05,128.16);(141.05,128.16)"
%	 stroke-width= "0.5"
%	 />
%</jpic>
%%End JPIC-XML
%PSTricks content-type (pstricks.sty package needed)
%Add \usepackage{pstricks} in the preamble of your LaTeX file
%You can rescale the whole picture (to 80% for instance) by using the command \def\JPicScale{0.8}
\ifx\JPicScale\undefined\def\JPicScale{1}\fi
\psset{unit=\JPicScale mm}
\psset{linewidth=0.3,dotsep=1,hatchwidth=0.3,hatchsep=1.5,shadowsize=1,dimen=middle}
\psset{dotsize=0.7 2.5,dotscale=1 1,fillcolor=black}
\psset{arrowsize=1 2,arrowlength=1,arrowinset=0.25,tbarsize=0.7 5,bracketlength=0.15,rbracketlength=0.15}
\begin{pspicture}(0,0)(266.12,158.16)
\newrgbcolor{userFillColour}{0.8 0.8 0.8}
\newrgbcolor{userShadowColour}{0.4 0.4 0.4}
\rput{0}(21.05,58.16){\psellipse[fillcolor=userFillColour,fillstyle=solid,shadow=true,shadowcolor=userShadowColour](0,0)(20,-10)}
\rput(21.05,58.16){4228}
\newrgbcolor{userFillColour}{0.8 0.8 0.8}
\newrgbcolor{userShadowColour}{0.4 0.4 0.4}
\rput{0}(91.05,68.16){\psellipse[fillcolor=userFillColour,fillstyle=solid,shadow=true,shadowcolor=userShadowColour](0,0)(20,-10)}
\rput(91.05,68.16){4664}
\newrgbcolor{userFillColour}{0.8 0.8 0.8}
\newrgbcolor{userShadowColour}{0.4 0.4 0.4}
\rput{0}(51.05,98.16){\psellipse[fillcolor=userFillColour,fillstyle=solid,shadow=true,shadowcolor=userShadowColour](0,0)(20,-10)}
\rput(51.05,98.16){43FE}
\psline{<-}(40.39,89.74)(24.47,68.03)
\psline{<-}(64.74,90.86)(77.3,75.39)
\newrgbcolor{userFillColour}{1 0.8 0.8}
\pspolygon[fillcolor=userFillColour,fillstyle=solid,shadow=true,shadowcolor=black](33.16,33.16)
(33.16,3.16)
(8.16,3.16)
(8.16,28.16)
(13.16,33.16)
(28.16,33.16)(33.16,33.16)
\rput(20.79,23.16){Datei X}
\rput(20.79,12.63){4378}
\psline(8.16,28.16)
(13.16,28.16)(13.16,33.16)
\psline[linestyle=dashed,dash=1 1](21.05,48.16)(21.05,33.16)
\newrgbcolor{userFillColour}{0.8 0.8 0.8}
\newrgbcolor{userShadowColour}{0.4 0.4 0.4}
\rput{0}(151.05,123.16){\psellipse[fillcolor=userFillColour,fillstyle=solid,shadow=true,shadowcolor=userShadowColour](0,0)(20,-10)}
\rput(151.05,123.16){437A}
\newrgbcolor{userFillColour}{0.8 0.8 0.8}
\newrgbcolor{userShadowColour}{0.4 0.4 0.4}
\rput{0}(96.05,148.16){\psellipse[fillcolor=userFillColour,fillstyle=solid,shadow=true,shadowcolor=userShadowColour](0,0)(20,-10)}
\rput(96.05,148.16){4377}
\newrgbcolor{userFillColour}{0.8 0.8 0.8}
\newrgbcolor{userShadowColour}{0.4 0.4 0.4}
\rput{0}(161.05,88.16){\psellipse[fillcolor=userFillColour,fillstyle=solid,shadow=true,shadowcolor=userShadowColour](0,0)(20,-10)}
\rput(161.05,88.16){4361}
\newrgbcolor{userFillColour}{0.8 0.8 0.8}
\newrgbcolor{userShadowColour}{0.4 0.4 0.4}
\rput{0}(196.05,53.16){\psellipse[fillcolor=userFillColour,fillstyle=solid,shadow=true,shadowcolor=userShadowColour](0,0)(20,-10)}
\rput(196.05,53.16){4A6D}
\newrgbcolor{userFillColour}{0.8 0.8 0.8}
\newrgbcolor{userShadowColour}{0.4 0.4 0.4}
\rput{0}(136.05,43.16){\psellipse[fillcolor=userFillColour,fillstyle=solid,shadow=true,shadowcolor=userShadowColour](0,0)(20,-10)}
\rput(136.05,43.16){4B4F}
\newrgbcolor{userFillColour}{0.8 0.8 0.8}
\newrgbcolor{userShadowColour}{0.4 0.4 0.4}
\rput{0}(101.05,13.16){\psellipse[fillcolor=userFillColour,fillstyle=solid,shadow=true,shadowcolor=userShadowColour](0,0)(20,-10)}
\rput(101.05,13.16){E791}
\newrgbcolor{userFillColour}{0.8 0.8 0.8}
\newrgbcolor{userShadowColour}{0.4 0.4 0.4}
\rput{0}(181.05,18.16){\psellipse[fillcolor=userFillColour,fillstyle=solid,shadow=true,shadowcolor=userShadowColour](0,0)(20,-10)}
\rput(181.05,18.16){57EC}
\newrgbcolor{userFillColour}{0.8 0.8 0.8}
\newrgbcolor{userShadowColour}{0.4 0.4 0.4}
\rput{0}(236.05,13.16){\psellipse[fillcolor=userFillColour,fillstyle=solid,shadow=true,shadowcolor=userShadowColour](0,0)(20,-10)}
\rput(236.05,13.16){AA93}
\psline{<-}(131.78,120.33)(70.99,99.01)
\psline{<-}(113.68,143.36)(134.28,128.75)
\psline{<-}(151.05,113.16)(157.89,98.03)
\psline{<-}(156.05,78.16)(136.05,53.16)
\psline{<-}(120.33,36.64)(104.08,23.03)
\psline{<-}(176.18,81.51)(193.09,63.16)
\psline{<-}(188.42,43.88)(183.36,28.22)
\psline{<-}(205,44.14)(220.26,19.34)
\newrgbcolor{userFillColour}{1 0.8 0.8}
\pspolygon[fillcolor=userFillColour,fillstyle=solid,shadow=true,shadowcolor=black](266.12,67.76)
(266.12,37.76)
(241.12,37.76)
(241.12,62.76)
(246.12,67.76)
(261.12,67.76)(266.12,67.76)
\rput(253.75,57.76){Datei X}
\rput(253.75,47.24){4378}
\psline(241.12,62.76)
(246.12,62.76)(246.12,67.76)
\psline[linestyle=dashed,dash=1 1](253.95,37.57)(247.24,21.38)
\psbezier[linestyle=dashed,dash=1 1]{<-}(15.07,67.83)(15.39,87.96)(20.17,97.19)(30.99,98.62)
\psbezier[linestyle=dashed,dash=1 1]{<-}(10,66.84)(6.68,112.53)(43.08,129.94)(131.32,124.87)
\psbezier[linestyle=dashed,dash=1 1]{<-}(4.93,64.14)(4.43,117.98)(25.84,143.44)(76.32,149.01)
\psbezier[linestyle=dashed,dash=1 1]{<-}(239.67,23.16)(224.42,109.37)(187.28,147.25)(115.86,149.41)
\psbezier[linestyle=dashed,dash=1 1]{<-}(236.32,23.16)(222.32,77.41)(202.72,107.15)(170.99,122.3)
\psbezier[linestyle=dashed,dash=1 1]{<-}(232.11,22.89)(219.95,67.47)(204.59,86.97)(180.92,87.89)
\psbezier[linestyle=dashed,dash=1 1]{<-}(225,21.58)(221.5,34.84)(217.63,42.42)(212.11,46.84)
\psline[linestyle=dashed,dash=1 1]{->}(240,125)(265,125)
\rput(253,120){{\tiny Objektverweis}}
\psline{->}(240,140)(265,140)
\rput(252,135){{\tiny Tapestry-Link}}
\psline[linewidth=0.5,linecolor=red]{->}(240,155)(265,155)
\rput(252,150){{\tiny Publish-Weg}}
\psline[linewidth=0.5,linecolor=red]{<-}(46.84,93.68)(25.26,63.95)
\psline[linewidth=0.5,linecolor=red]{<-}(136.58,125.26)(61.32,99.21)
\psline[linewidth=0.5,linecolor=red]{<-}(205.53,49.74)(225.53,17.11)
\psline[linewidth=0.5,linecolor=red]{<-}(168.68,84.21)(191.32,59.47)
\psline[linewidth=0.5,linecolor=red]{<-}(152.11,118.16)(162.89,94.74)
\psline[linewidth=0.5,linecolor=red]{<-}(110.79,149.47)(141.05,128.16)
\end{pspicture}

  \caption[Tapestry: Veröffentlichen eines Objekts]{Veröffentlichen des Objekts
  \texttt{4378}. Vgl. \cite[Fig. 4]{TapestryTR}}
  \label{fig:tapestry-publish-object}
\end{figure}

Im Beispiel (Abb. \ref{fig:tapestry-publish-object}) haben sowohl der Knoten 
\texttt{4228} als auch der Knoten \texttt{AA93} eine Kopie des Objekts (hier 
eine Datei) \texttt{4378} bereitgestellt. Beide senden nun die 
\textsl{publish}-Nachricht an den Root-Knoten von \texttt{4378}, in diesem 
Beispiel der Knoten mit der ID \texttt{4377}. Wie aufgezeigt, speichern alle 
Peers, die die Nachricht weiterleiten, einen lokalen Objektverweis. Hier sieht 
man auch, dass z.B. Knoten \texttt{437A} jeweils einen Verweis auf beide 
bereitstellende Knoten hält.

\subsection{Surrogate-Routing} \label{subsec:surrogate-routing}
Dieses Verfahren bietet Abhilfe für das Problem der möglichen Löcher in der
Routingtabelle. Diese können z.B. dann auftreten, wenn die Abbildungsfunktion
$MapRoots()$ nicht-existierende Knoten zurückliefert.

Die Lösung besteht nun darin, dass Routing ganz normal Richtung Root-Knoten 
laufen zu lassen. Trifft man nun auf ein Loch in der Routingtabelle, so wird 
versucht, die Nachricht an einen anderen Nachbarn in einem höheren zu senden. 
Schlägt auch dies fehl, so wird der nächst höhere Level versucht, usw. Aufgrund 
der Eigenschaften des Tapestry-Netzwerks lässt sich durch das Surrogate-Routing 
stets ein eindeutiger Knoten erreichen, unabhängig von der Startposition des 
Routings. Einen Beweis für diese Annahme findet sich in \cite{HildrumKRZ02} bzw. 
\cite{Schindelhauer2004}.

\section{Organisation}
\subsection{Einfügen eines Knotens}
Der Eintritt eines neuen Knotens $N$ in das Tapestry-Netzwerk erfolgt in drei
Schritten:

\begin{enumerate}
  \item Senden einer Nachricht an einen Knoten mit der eigenen ID. Dadurch wird
  aufgrund des Surrogate-Routings der Surrogate-Knotens $S$ von $N$ gefunden.
  Dieser Knoten hat eine ID, die numerisch nahe zur eigenen ID liegt. Die
  Nachbarschaftstabellen von $S$ werden zu $N$ übernommen.
  \item $S$ bestimmt $p$, die Länge des gemeinsamen Präfixes von $S_{id}$ und
  $N{_id}$. $S$ sendet nun einen sog. \textsl{Acknowledged Multicast} an
  diejenigen Knoten, deren ID in der selben Präfixlänge $p$ mit $N_{id}$
  übereinstimmt. Die so erreichten Knoten fügen $N$ in ihre Routingtabellen ein
  und übertragen im Gegenzug wenn nötig Objektverweise an $N$, da dieser neue
  Knoten ja möglicherweise der neue Root-Knoten eines Objekts werden könnte.
  \item Zunächst erstellt $N$ seine Nachbarschaftsmenge aus denjenigen Knoten,
  die auf den \textsl{Acknowledged Multicast} geantwortet haben. Um jedoch
  eine optimierte Routingtabelle zu erhalten und die Konsistenz- sowie
  Lokalitätseigeschaften (s. \cite{HildrumKRZ02}) zu gewährleisten, wird im
  folgenden iterativ eine sog. \textsl{nearest neighbor search} durchgeführt.
\end{enumerate}

Nach Durchführung dieser Schritt ist $N$ in das Tapestry-Netzwerk integriert -
es besteht kein Unterschied zu dem Netzwerk das entstanden wäre, wenn $N$
bereits von Anfang an Bestandteil gewesen wäre.

\subsection{Austritt eines Knotens}
Im Gegensatz zu Pastry unterscheidet Tapestry zwischen einem gewollten und einem
ungewollten Austritt von Knoten aus dem Netzwerk.

\subsubsection{Freiwilliger Austritt}
Ein freiwilliger Austritt aus dem Netz kann in Tapestry relativ einfach
gehandhabt werden. Denn wie beschrieben, hat ein Knoten $N$ jeweils eine Liste
derjenigen Knoten gespeichert, die auf ihn verweisen (Backpointers). Diese
Knoten kann er nun beim Verlassen des Netzes über seine Absicht informieren und
gleichzeitig einen Ersatzknoten für jedes Level aus seiner Routingtabelle
mitliefern - auf diese Weise kann das entstehende Loch im Netz recht schnell
geschlossen werden.

Um Objekte, die auf dem zu löschenden Knoten $N$ liegen, weiterhin zugänglich 
zu machen, republiziert $N$ die auf ihm gespeicherten Objekte und leitet bis 
zur abgeschlossenen Abmeldung eingehende Anfragen an lokale Objekt an den neuen 
Root-Knoten weiter.

\subsubsection{Unfreiwilliger Austritt}
Grundsätzlich schwieriger ist ein unerwarteter Austritt eines Tapestry-Knotens.
Dieser Fall ist jedoch aufgrund der Fehleranfälligkeit des Internets der weitaus
häufigere. Tapestry begegnet diesem Problem zum einen durch redundante
Routingtabellen- und Objektzeigereinträge.

Zur Erhaltung des Netzes schicken die Knoten weiterhin periodische Nachrichten 
an die Nachbarn, um so nicht mehr antwortende Knoten zu erkennen. Tritt ein 
solcher Fehlerfall auf, so wird ein Recovery-Mechanismus aktiviert, der von den 
Autoren des Papers jedoch nicht näher beschrieben wird. Nach 
\cite[S.~44]{TapestryTR} bleibt jedoch eine nahezu 100\%ige Funktionsfähigkeit 
des Netzes gegeben, für den Fall, dass nicht pausenlos Knoten ausfallen.