Es soll ein neuronales Netz entworfen werden, welches in der Lage ist zu
entscheiden, ob bei einem Patienten eine Fehlfunktion der Schilddrüse vorliegt.
Dazu müssen die Werte 21 verschiedener Attribute ausgewertet werden. Anhand der
hohen Anzahl der Attribute lässt sich erkennen, dass eine solche
\emph{Klassifizierungsaufgabe} mit herkömmlichen Methoden eher schwierig lösbar
wäre. Daher wird hier auf die Verwendung eines künstlichen neuronalen Netzes
zurückgegriffen, welche sich gut zur Klassifikation und Mustererkennung nutzen
lassen.

In der vorliegenden Arbeit sollen zunächst die zur Verfügung stehenden Daten
analysiert, in Test- und Trainingsdaten aufgeteilt und anschließend ein
entsprechendes neuronales Netz mithilfe der \emph{Neural Networks Toolbox} der
Mathematiksoftware Matlab implementiert werden. Ziel ist es, eine
Klassifizierungsrate zu erreichen, die deutlich über 92\% liegt.

