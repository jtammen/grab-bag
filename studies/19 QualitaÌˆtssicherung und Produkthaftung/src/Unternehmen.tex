Als Beispiel soll in dieser Ausarbeitung die fiktive Software-Firma 
\emph{foobar GmbH} dienen. Es handelt sich dabei um eine mittelständische Forma 
mit 20 Angestellten. Zu ihren Kunden gehören hauptsächlich mittelgroße und große
Verlage, Versicherungen und Banken. Für diese Kunden werden
Individualsoftwaresysteme konzipiert, erstellt und betreut.

Die Struktur des Unternehmens lässt sich wie folgt in mehrere Bereiche
aufteilen: 

\begin{description}
  \item[Marketing und Vertrieb:] 2 Mitarbeiter. Diese Abteilung ist
  verantwortlich für die Darstellung des Unternehmens nach außen sowie die
  Akquise neuer sowie Betreuung bestehender Kunden. So werden z.B. Werbe- oder
  Stellenanzeigen geschaltet und Messebesuche durchgeführt.
  \item[Entwicklung:] 14 Mitarbeiter. Die größte Abteilung der Firma kümmert
  sich um das Design und die Entwicklung der Kundenprojekte. Das Entwicklerteam
  besteht dabei komplett aus Hochschulabsolventen und es arbeiten Mitarbeiter
  unterschiedlicher Erfahrungsgrade zusammen. Die Entwicklungsabteilung ist
  aufgeteilt auf 2 Projektteams, die wiederum durch jeweils einen Projektleiter
  nach außen hin vertreten werden.
  \item[Support:] 2 Mitarbeiter. In dieser Abteilung werden alle dringenden
  Kundenanfragen entgegengenommen und bearbeitet. Sofern es sich nicht um
  allgemein lösbare Probleme handelt, werden diese je nach Schweregrad sofort an
  das zuständige Entwicklerteam weitergeleitet.
  \item[Verwaltung:] 1 Mitarbeiter. Zu den Aufgaben dieser Abteilung gehört die
  Terminverwaltung, Bestellung von Bürdobedarf sowie die Buchhaltung.
\end{description}

Da es sich bei den Kunden teils um Unternehmen aus der Finanzwirtschaft 
handelt, bestehen natürlich erhöhte Anforderungen an die Qualität der von der 
Firma erstellten Produkte.