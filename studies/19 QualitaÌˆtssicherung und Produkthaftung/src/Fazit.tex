Die Einführung eines Qualitätsmanagementsystems kann nur dann erfoglreich sein, 
wenn der Prozess durch das gesamte Unternehmen vorangetrieben wird. Die 
Mitarbeiter müssen die Qualitätsansprüche und die Kundenorientierung täglich in 
ihrer Arbeit umsetzen, andernfalls bringt ein QMS langfristig gesehen keine
Vorteile. 

Durch die Einführung des QMS erhofft sich die \emph{foobar GmbH} einen eher
langfristig sicht- und messbaren Erfolg. Durch die Einhaltung von
Qualitätsstandards sollen die Abläufe optimiert, Kosten reduziert und so
letztendlich auch die Attraktivität der Firma für potentielle Neukunden erhöht
werden. 

Auch wenn die Einführung eines QMS zunächst einige Zeit und auch finanzielle
Mittel beansprucht, so lohnt es sich meiner Meinung auf jeden Fall, diesen
Schritt zu gehen und sich für Qualität einzusetzen.