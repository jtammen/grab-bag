Nach Durchführung einiger Testläufe mit dem Demoskript \texttt{demotsplib.m} 
und dem TSP-Problem \texttt{bays29.tsp} zeigte sich, dass die Ergebnisse i.A. 
noch recht weit vom Optimalwert 2020 (vgl. \cite[Appendix]{Reinelt94}) entfernt 
waren.

Das Beispielskript ist standardmäßig mit den folgenden Parametern konfiguriert:

\begin{itemize}
  \item \texttt{NumberSubpopulation}: Anzahl der Subpopulationen, 6
  \item \texttt{NumberIndividuals}: Anzahl der Individuen pro Population, 50
  \item \texttt{Selection.GenerationGap}: Anzahl der Individuen pro Population, 0.95
  \item \texttt{Termination.Method}: Art des Abbruchkriteriums, 1 (entspricht
  dem Erreichen der maximalen Anzahl von Generationen)
  \item \texttt{Termination.MaxGen}: Anzahl der Generationen, nach denen der
  Algorithmus terminieren soll, 400
\end{itemize}

Durch die Wahl des genetischen Algorithmus' \texttt{tbx3perm} sind implizit 
weiterhin die folgenden Einstellungen gegeben:

\begin{itemize}
  \item \texttt{VariableFormat}: Format der Variablen und Konvertierung 
  zwischen diesem und in dem intern verwendeten Format, 5 (entspricht einem 
  nicht festgelegten Format, demzufolge findet auch keine Konvertierung statt).
  \item \texttt{Recombination.Name}: Verwendetes Rekombinationsverfahren, 
  \texttt{recpm} (entspricht Partial Matching Recombination)
  \item \texttt{Mutation.Name}: Verwendetes Mutationsverfahren, \texttt{mutswap}
  (entspricht Swap Mutation)
\end{itemize}