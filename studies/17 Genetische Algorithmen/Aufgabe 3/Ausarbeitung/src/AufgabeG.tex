\subsection{Optimale Parameter}
Durch die in dieser Arbeit durchgeführten Testläufe ergeben sich für uns die
optimalen Parameter wie folgt:

\begin{description}
  \item[Anzahl Generationen:] 100
  \item[Anzahl Subpopulationen:] 12
  \item[Anzahl Individuen:] 60
  \item[Selektionsoperator:] Tournament Selection, \texttt{seltour}
  \item[Rekombinationsoperator:] Recombination Partial Matching, \texttt{recpm}
  \item[Mutationsoperator:] Kombination aus \texttt{mutswap}, \texttt{mutmove} 
	und \texttt{mutinvert}
\end{description}

\subsection{Testläufe mit variierenden TSP-Problemen}
Abschließend führen wir jeweils 20 Testläufe mit den o.g. Parametern und drei
verschiedenen TSP-Problemen durch; die Ergebnisse sind in Tabelle
\ref{tbl:aufgabeG-ergebnisse} aufgezeigt.

\begin{table}
	\sffamily
	\centering
	\footnotesize
	
	\begin{threeparttable}
	\begin{tabularx}{\textwidth}{TXlllp{7em}p{7em}}
		\toprule
		\multicolumn{1}{@{}N}{TSP-Bsp.} &
		\multicolumn{1}{V{3em}@{}}{Subpop.} &
		\multicolumn{1}{V{3em}@{}}{Indiv.} &
		\multicolumn{1}{V{5em}@{}}{Mittelwert $\bar{x}$} &
		\multicolumn{1}{V{6.5em}@{}}{Standardabw. $\sigma_x$} &
		\multicolumn{1}{V{8em}@{}}{Minimalwert in Lauf $r$, Generation $g$} &
		\multicolumn{1}{V{8em}@{}}{Maximalwert in Lauf $r$, Generation $g$} \\
		\midrule\addlinespace
		bays29 & 12 & 60 & 2140,65 & 87,39 & 2028, $r = 13$, $g = 96$ & 2375, $r = 6$, $g = 95$ \\
		\midrule
		bayg29 & 12 & 60 & 1692,90 & 39,54 & 1610, $r = 16$, $g = 89$ & 1746, $r = 13$, $g = 97$ \\
		\midrule
		berlin52 & 12 & 60 & 11218,55 & 660,23 & 9854, $r = 20$, $g = 100$ & 12326, $r = 2$, $g = 99$ \\ \cmidrule(rl){1-7}
		berlin52\tnote{1} & 12 & 60 & 8262,10& 278,77 & 7786, $r = 18$, $g = 362$ & 8931, $r = 7$, $g = 394$ \\ \cmidrule(rl){1-7}
		berlin52\tnote{1} & 20 & 100 & \textbf{8110,45} & \textbf{193,89} & \textbf{7734}, $r = 1$, $g = 336$ & 8424, $r = 8$, $g = 330$ \\

		\addlinespace\bottomrule
	\end{tabularx}
	\begin{tablenotes}
	  \footnotesize\normalfont
    	\item [1] Anzahl der Generationen: 400
 	\end{tablenotes}
	\end{threeparttable}
	\caption{Ergebnisse der Testreihe}
	\label{tbl:aufgabeG-ergebnisse}
\end{table}

\subsubsection{Interpretation der Ergebnisse}
Der erreichte Wert für \texttt{bays29} erschien uns mit 2028 nahe genug am
Idealwert 2020. Für \texttt{bayg29} konnte sogar der bestbekannte Wert von 1610
(vgl. \cite[Appendix]{Reinelt94}) erreicht werden.
Für diese beiden TSP-Beispiele scheinen unsere gefunden Einstellungen demnach
sehr gut zu funktionieren. Im Gegensatz dazu verfehlten wir bei dem Problem mit
52 Knoten den Optimalwert 7542 um Längen. Daher versuchten wir, durch
Heraufsetzen der Generationen hier noch Verbesserungen zu erlangen, was uns in
Maßen auch gelang. Durch Einsatz von 400 Generationen wurden die Ergebnisse
deutlich verbessert, wie die Tabelle zeigt. 

Durch die Erhöhung der Subpopulationen sowie der Anzahl der Individuen im letzten
Testlauf konnten weitere leichte Verbesserungen erreicht werden. So ergibt sich
der von uns erreichte Minimalwert zu 7734.
