\subsection{Architektur des Testsystems}
\label{subsec:ArchitekturTestsystem}
Für die Durchführung der Tests wurde zunächst das der GEATbx beiliegende Skript
\textbf{demotsplib.m} als Ausgangsbasis benutzt. Alle für die Testläufe zu
variierenden Parameter lassen sich im Kopf des Skriptes \textbf{tsp.m} in
Variablenform festlegen. 

Für die Durchführung der Aufgabe c) waren beispielsweise die Anzahl der
Subpopulationen und der Individuen pro Population in verschiedenen
Konstellationen zu testen. Zusätzlich dazu lassen wir für jede
Variablenkombination eine definierbare Anzahl von Testläufen laufen, um später
Mittelwert und Standardabweichung berechnen zu können. Listing
\ref{lst:testsystem-pseudo} zeigt den beispielhaften Aufbau des Testskripts für
Aufgabe c). Für Aufgabe d) wurde eine modifizierte Version des in
\ref{lst:testsystem-pseudo} beschriebenen Skriptes eingesetzt. Der Pseudocode
zu der genannten Modifikation wird in Listing \ref{lst:testsystemTeilD}
aufgeführt. Der komplette Code der Skripte \ref{lst:testsystem-pseudo} und
\ref{lst:testsystemTeilD} ist im Anhang aufgeführt.

\begin{lstlisting}[language=Pseudo, label={lst:testsystem-pseudo}, 
caption={Pseudocode Testskript Teil B}]
Subpopulationen = (1, 5, 8, 12)
Individuen = (15, 30, 45, 60)
AnzahlTestlaeufe = 20

foreach anzahlSubpopulationen in Subpopulationen do
	foreach anzahlIndividuen in Individuen do
		for <AnzahlTestlaeufe> do
			Fuehre genetischen Algorithmus aus
			Berechne Minimal-/Maximalwert
		end for
		Ergebnisausgabe Logfile
	end foreach
end foreach
\end{lstlisting}

\begin{lstlisting}[language=Pseudo, label={lst:testsystemTeilD}, 
caption={Pseudocode Testskript Teil D}]
Subpopulationen  = 12;
Individuen 	     = 60;
AnzahlTestlaeufe = 20;

SelektionsMethoden = (selsus, selrws, seltour)

foreach SelektionsMethode in SelektionsMethoden do
	for <AnzahlTestlaeufe> do
			Fuehre genetischen Algorithmus aus
			Berechne Minimal-/Maximalwert
	end for
		Ergebnisausgabe Logfile
end foreach
\end{lstlisting}

Zur späteren Auswertung der Ergebnisse werden alle relevanten Daten in ein
Logfile protokolliert. Abbruchkriterium für die genetische Optimierung ist hier
das Erreichen der Maximalgeneration 100.

\subsubsection{Eingesetzte Hard- und Software}
Die Tests der Aufgabe c) wurden auf einem Pentium M mit 1.6 GHz unter 
Verwendung von Matlab 5.3 und GEATbx Version 3.8 durchgeführt.