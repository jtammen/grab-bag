\section{Was versteht man unter nichtrealistischer Computergraphik?}
Unter \textbf{Nichtrealistischer Computergraphik (non photorealistic rendering – 
kurz NPR)} versteht man die computergestützte Herstellung/Erzeugung von Bildern
aus Computerdaten. \par Diese Verfahren stellen das Gegenteil zu 
(hyper-)realistischen 2D/3D-Computer\-ani\-ma\-tio\-nen dar. Während es Ziel der 
meisten Algorithmen zum Rendern von Computergrafiken ist, ein Ergebnis zu 
erzeugen, welches möglichst nahe an die Qualität eines Fotos gelangt, wird 
diese Beschränkung bei NPR aufgehoben. Mithilfe derzeitiger Algorithmen ist es 
so, ausgehend von denselben dreidimensionalen Modellen beispielsweise möglich, 
Bilder zu erzeugen, die wie Strichzeichnungen, Aquarelle, Ölgemälde o.ä. 
aussehen. Diese Art von Algorithmen, die geometrisch korrekte Formen nicht 
fotorealistisch darstellen, sind am häufigsten vertreten und (zumeist) auch als 
rein bildbasierte Varianten (also ohne Betrachtung der zugrundeliegenden 
Geometrie) möglich. Es gibt jedoch auch weitere Bereiche des NPR wie z.B. nicht 
fotorealistische Beleuchtung oder nicht fotorealistische Projektion: hierbei 
wird mit optischen Gesetzen gebrochen und so Darstellungen ermöglicht, die in 
der Realität physikalisch unmöglich sind. Beispiele hierfür sind inverse 
Perspektive (parallele Linien divergieren, Objekte werden mit zunehmender 
Entfernung von der Kamera größer), mehrere Kamerastandpunkte, die in einem Bild 
vereint sind, nicht planare Projektionsflächen u.a.

\section{Motivation - Warum NPR?}
Seit über 40 Jahren ist das Hauptziel der Computergraphik die Erzeugung 
photorealistischer Bilder. Hierbei verwendet man so weit wie möglich:
\begin{itemize}
  \item realistische 3D-Modelle,
  \item realistische Beleuchtungssimulationen und
  \item realistische Farbwertzuweisungen.
\end{itemize}
Die Intention besteht darin, eine große Detailtreue in der Simulation der 
Realität zu erreichen, so dass die Bilder von Photographien nicht mehr zu 
unterscheiden sind. Hierbei treten folgende Probleme auf:
\begin{itemize}
  \item Die Erzeugung dieser Bilder ist sehr zeitaufwendig
  \item Die Bilder sind langsam und kostenintensiv im Druck
  \item Sie sind zuweilen nicht einfach zu interpretieren, da wichtige Details 
  nicht betont werden
  \item Das Anliegen eines Bildautors und damit die Berücksichtigung des 
  kommunikativen Ziels, das mit einem Bild verbunden wird, spielt in der Regel 
  keine Rolle
\end{itemize}
  Bei traditionellen Printmedien lässt sich folgendes feststellen:
\begin{itemize}
  \item Fotos spielen eine untergeordnete Rolle
  \item In den meisten Fällen werden abstraktere Darstellungsformen wie 
  Skizzen, Illustrationen oder Schemazeichnungen verwendet. Dabei ergeben sich 
  folgende Vorteile:
  \begin{itemize}
    \item Druck und Photokopie abstrakter Bilder sind schnell und preiswert
    \item Die Bilder sind einfacher und beschränken sich auf das Wesentliche
    \item Sie können wichtige Eigenschaften hervorheben und damit die 
    Aufmerksamkeit des Betrachters gezielt steuern
    \item Sie lassen sich einfach mit textuellen Darstellungen kombinieren
  \end{itemize}
\end{itemize}
Seit Ende der 80er Jahre werden verstärkt mit dem Computer auch abstrakte 
graphische Darstellungen erzeugt. Hierfür wird das Schlagwort 
"`Nicht-Photorealistische Computergraphik"' genutzt. Deussen 
\cite{Deussen2001} schlägt dagegen den Begriff "`Generalisierte graphische 
Darstellungen"' vor, als eine Erweiterung der engen Fokussierung auf den 
Photorealismus. \par Ein großer Vorteil dieser NPR-Technik gegenüber der 
Fotoerzeugung bzw. Fotodarstellung besteht darin, dass bestimmte 
ausgewählte, visuelle Informationen explizit dargestellt werden können. 
Diese Darstellungsart wird hauptsächlich im Bereich der Printmedien 
(Skizzen, Illustrationen, Schemazeichnungen, ...) oder in der Kunst 
(Cartoons, Farbverfälschungen, Nachempfinden von Malstilen,...) eingesetzt. 
\par Das Hauptziel hierbei ist es also, im Gegensatz zur realitätsnahen 
Bilddarstellung, durch die Anwendung verschiedener Darstellungsstile die 
Aussagekraft von Bildern zu erhöhen.
\par Erreicht werden diese Ziele durch die Techniken die im nächsten Abschnitt
vorgestellt werden.
