\section{Wasserfarben}
\subsection{Einführung}
\begin{frame}{Einführung Wasserfarben}
  \begin{itemize}
    \item Moderne Wasserfarbentradition: zweite Hälfte des 18. Jahrhunderts
    \item Ermöglicht zahlreiche künstlerische Effekte
    \item Computer-Erzeugung durch zwei Ansätze (nach Curtis et al.): 
    realistische physikalische Simulation (hoher Aufwand) oder Annäherung durch 
    geeignete Filter (echtzeitgeeignet)
  \end{itemize}
\end{frame}

\begin{frame}{Materialien}
  \begin{itemize} 
    \item Papier: meist nicht aus Zellstoff, sondern Leinen bzw.\ Baumwolle;
    grobe Struktur, dadurch Wasserabsorption und Diffusion möglich
    \item Bestandteile der Wasserfarbe: in Wasser gelöstes Pigment, Bindemittel
    sowie Tensid (oberflächenaktiver Stoff)
    \item Pigment: festes Material, besteht aus einzelnen Partikeln; oftmals in
    Puderform; kann in das Papier einsickern und dort anhaften bzw.\ durch das
    Wasser transportiert werden
    \item Bindemittel: Pigment kann an das Papier anhaften
    \item Tensid: Wasser kann in das Papier eindringen
  \end{itemize}
\end{frame}

\begin{frame}{Haupt-Pinseltechniken}
  \begin{itemize}
    \item "`wet-on-wet painting"': Malen mit nassem, farbgetränkten Pinsel auf
    nassem Papier
    \item "`wet-on-dry painting"': Malen auf trockenem Papier
  \end{itemize}
\end{frame}

\begin{frame}{Effekte}
  \centering\pgfimage[width=1.0\textwidth]
    {../images/Curtis1997-real-watercolor-effects}
  
  \begin{itemize}
    \item \textsl{Dry-brush}: trockener Pinsel/Papier; unregelmäßige Lücken
    \item \textsl{Edge darkening}: Pigmente von innen n.\ außen; dunkle Kanten
    \item \textsl{Backruns}: Wasser läuft in nasse Region zurück
    \item \textsl{Granulation} u.\ \textsl{Separation}: körnige Struktur, Farbtrennung
    \item \textsl{Flow patterns}: freies Ausbreiten der Pinselstriche
    \item \textsl{Color glazing}: Übereinanderlegen von dünnen Farbschichten
  \end{itemize}
\end{frame}

\subsection{Computererzeugte Wasserfarben}
\begin{frame}{Simulation}
  \begin{itemize}
    \item Wichtig: nicht nur physikalische Eigenschaften, sondern auch
    künstlerische Effekte berücksichtigen!
    \item Bestandteile: Papierschicht, geordnete Menge von Farbschichten
    \item Für jede Farbschicht wird Flüssigkeitssimulation separat durchgeführt
    \item Zusätzlich: "`wet-area"'-Maske
    \item Nach der Simulation: Komposition mit Kubelka-Munk-Farbmodell erzeugt 
    finale Darstellung
  \end{itemize}
\end{frame}

\begin{frame}{Simulation: Papier}
  \begin{itemize}
    \item Vereinfachtes Papier-Modell: Textur wird als Höhenfeld und
    Flüssigkeitskapazitätsfeld abgebildet
    \item Höhe: $0 < h < 1$
    \item Steigung wird zur Berechnung der Fließgeschwindigkeit in der
    Simulation benutzt
    \item Kapazität: $c = h(c_{max} - c_{min}) + c_{min}$
  \end{itemize}
\end{frame}

\begin{frame}{Simulation: 3-Schichten-Modell}
  \begin{columns}
    \column{.6\textwidth}
	    \begin{enumerate}
	      \item "`shallow-water layer"': Wasser fließt über die Oberfläche und
	      transportiert ggf.\ Pigmente
	      \item "`pigment-deposition layer"': Pigmente werden hier abgelagert und
	      von hier weitertransportiert
	      \item "`capillary layer"': Absorbiertes Wasser diffundiert durch
	      Kapillarwirkung
	    \end{enumerate}

	\column{.4\textwidth}
		\pgfimage<3->[width=0.5\textwidth]{../images/Curtis1997-layer-shallow-water}\par
		\pgfimage<4->[width=0.5\textwidth]{../images/Curtis1997-layer-pigment-deposition}\par
		\pgfimage<5->[width=0.5\textwidth]{../images/Curtis1997-layer-capillary}
  \end{columns}
\end{frame}

\begin{frame}{Simulation: Haupt-Schleife}
  Eingabewerte: (Anfangswerte)
  \begin{itemize}
    \item Wet-Area-Maske $M$
    \item Geschwindigkeit des Wassers $u$ und $v$
    \item Wasserdruck $p$
    \item Pigmentkonzentration $g^k$
    \item Wassersättigung des Papiers $s$
  \end{itemize}
  
  \pause
  Simuliert werden über einen definierten Zeitraum: Wasser- und Pigmentbewegungen,
  Pigmentablagerungen, Kapillarfluss.
\end{frame}

\begin{frame}{Simulation: Rendering und Pigmente}
  \begin{itemize}
    \item Verwendung des Kubelka-Munk-Farbmodells
    \item Jedem RGB-Farbkanal jeden Pigments sind zwei Koeff. zugewiesen:
    \begin{itemize}
      \item Absorptionskoeff. $K$
      \item Streuungskoeff. $S$
    \end{itemize}
  \end{itemize}
  
  \begin{block}{Synthetische Pigmente}
    \pgfimage<5->[width=0.8\textwidth]{../images/Curtis1997-pigments}
  \end{block}
\end{frame}

\begin{frame}{Simulation: Vergleich der Ergebnisse}
  \centering\pgfimage[width=1.0\textwidth]
  {../images/Curtis1997-real-watercolor-effects}\par
  \centering\pgfimage[width=1.0\textwidth]
  {../images/Curtis1997-simulated-watercolor-effects}
\end{frame}

\begin{frame}{Anwendungen}
  \begin{itemize}
    \item Interaktives Malen: Benutzer kann Startzustand der Simulation
    "`malen"' durch Erstellung der Schichten mit Unterschichten für Pigmente,
    Wasser und die Wet-Area-Maske. Zusätzlich: globales Referenzbild sowie
    Papiertextur
    \item "`Watercolorization"': Konvertierung eines Farbbilds in eine
    Wasserfarben-Illustration.
    \item 3D-Szenen: Erweiterung der "`Watercolorization"'
    \item Aufbauende Ansätze: Darstellung in Echtzeit
  \end{itemize}
\end{frame}

\subsection{Echtzeit-Wasserfarben-Animationen}
\begin{frame}{Echtzeit-Animationen}
  \begin{itemize}
    \item Ansatz nach O. Deussen und T. Luft: keine exakte physikalische 
    Simulation, sondern Mittelweg zwischen Qualität und Rechenaufwand mit Ziel: 
    Echtzeit-Rendering von 3D-Szenen
    \item Aspekte: \textbf{Abstraktion und Vereinfachung},
    \textbf{Wasserfarben-Effekte}, \textbf{Licht und Schatten}
    \item Grobe Vorgehensweise: Erzeugung einzelner abstrakter 
    Wasserfarben-Layer, die mithilfe der Grafikkarte zusammengefügt werden
  \end{itemize}
\end{frame}

\subsubsection{Vorgehensweise}
\begin{frame}{Abstraktion und Vereinfachung}
  \begin{columns}[t]
    \column{0.6\textwidth}
      \begin{itemize}
        \item Jeder Layer enthält jeweils ein oder mehrere gleichartige Objekte
        \item Segmentieren der 3D-Szene anhand eindeutiger Identifikatoren
        (Ergebnis: "`intensity images"')
        \item Low-Pass-Filter erzeugt abstrakte, weiche Formen mit definierbarem
        Detaillierungsgrad
        \item Layer enthalten außerdem Farb- sowie Transparenzinformation
      \end{itemize}
    \column[T]{0.4\textwidth}
      \pgfimage<4->[width=1.0\textwidth]{../images/Luft2006-tree-segmentation}
  \end{columns}
\end{frame}

\begin{frame}{Formextraktion und Fließmuster}
  \begin{columns}[t]
    \column{0.6\textwidth}
      \begin{itemize}
        \item Form ergibt sich aus den Intensitäts-Werten der \textsl{intensity
        images}
        \item Fließmuster beschreibt das Pigment-Verhalten am Rand des Layers; 
        \textsl{wet-on-dry}: harte Kanten, \textsl{wet-on-wet}: weiche, 
        federartige Kanten
      \end{itemize}
    \column[T]{0.4\textwidth}
      \pgfimage<4->[width=1.0\textwidth]{../images/Luft2006-tree-shape-extraction-flow-pattern}
  \end{columns}
\end{frame}

\begin{frame}[label=current]{Edge darkening}
  \begin{columns}[t]
    \column{0.6\textwidth}
      \begin{itemize}
        \item Pigmente werden beim Trocknen zum Rand transportiert
        \item Imitation durch Anwendung eines Gauss-Filters, erzeugt weichen
        Intensitäts-Übergang an den Rändern
        \item Erzeugung eines Gradienten, der dem Wasserfarbenfluss folgt und
        die Transparenz an den Rändern verblassen lässt
      \end{itemize}
    \column[T]{0.4\textwidth}
      \pgfimage<4->[width=0.8\textwidth]{../images/Luft2006-tree-edge-darkening}
  \end{columns}
\end{frame}

\begin{frame}{Pigment-Granulation und Transparenz}
  \begin{columns}[t]
    \column{0.6\textwidth}
      \begin{itemize}
        \item Struktur des Papiers beeinflusst Wasserverlauf und erzeugt so Pigment-Granulation
        \item Imitation durch Verwendung einer zusätzlichen Papier-Textur
        \item Mit entsprechenden Texturen lassen sich auch ausgefranste Kanten erzeugen
      \end{itemize}
    \column[T]{0.4\textwidth}
      \pgfimage<3->[width=1.0\textwidth]{../images/Luft2006-tree-granulation-transparency}
  \end{columns}
\end{frame}

\subsubsection{Komposition von Wasserfarben-Layern}
\begin{frame}{Licht (1)}
  \begin{columns}
    \column{0.6\textwidth}
      \begin{itemize}
        \item Verwendung der Beleuchtungs-Informationen aus dem 3D-Modell zur
        Anpassung der erzeugten bzw.\ zur Erstellung neuer Layer
        \item Phong-Beleuchungsmodell, Komponenten: ambiente (hier konstant), 
        diffuse und spiegelnde Reflexion
        \item Diffuse und spiegelnde Reflexion werden auf zwei \textsl{lighting-maps} gerendert
      \end{itemize}
    \column{0.4\textwidth}
      \pgfimage<4->[width=1.0\textwidth]{../images/Luft2006-still-life-light-shadow}
  \end{columns}
\end{frame}

\begin{frame}{Licht (2)}
  \begin{columns}
    \column{0.6\textwidth}
      \begin{itemize}
        \item spiegelnde Reflexion: erzeugt hervorgehobene Region, wird erreicht
        durch Ausmaskieren der \textsl{intensity-images} vor Erstellung des
        Layers (Bild 3 und 4)
        \item diffuse Reflexion: Erzeugung zusätzlicher Layer, z.\,B.\
        Hervorhebung dunkler Stellen (Bild 3); weiterhin Anpassung aller Layer,
        zwei Farbwerte für unbeleuchtete und beleuchtete Regionen (Bild 2)
      \end{itemize}
    \column{0.4\textwidth}
      \pgfimage<2->[width=1.0\textwidth]{../images/Luft2006-still-life-light-shadow}
  \end{columns}
\end{frame}

\begin{frame}{Komposition}
  \begin{columns}[t]
    \column{0.6\textwidth}
      \begin{itemize}
        \item Layer werden mit Standard-Funktion zur Überblendung
        transparenter Objekte zusammengesetzt
        \item Farbe an der Position $(x,y)$ ergibt sich zu: $R_{rgb} = C_a \cdot
        C_{rgb} + (1-C_a) \cdot B_{rgb}$
        \item $C_{rgb}$ = Farbe, $C_a$ = Transparenz, $B_{rgb}$ = Hintergrundfarbe
      \end{itemize}
    \column[T]{0.4\textwidth}
      \pgfimage<4->[width=1.0\textwidth]{../images/Luft2006-tree-result}
  \end{columns}
\end{frame}

\begin{frame}{Demo}
  \begin{center}
	  \movie[autostart, height=6cm, width=8cm]{}{../tree_cover.avi}
  \end{center}
\end{frame}