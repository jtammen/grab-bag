\section{XML-Unterst�tzung}

F�r die Unterst�tzung von \acr{xml} bietet Oracle eine Reihe von Tools an. Diese Tools sind (Quelle: \citep{url:OracleXMLFAQ}):
\begin{itemize}
	\item XMLDB oder kurz XDB: Beinhaltet die mit Oracle 9i (oder sp�tere Versionen) standardm��ig ausgelieferte \acr{xml}-Unterst�tzung.
	\item \acr{xml}-\acr{sql}-Utility (XSU): Interfaces f�r die Programmiersprachen \acr{plsql} und Java.
	\item \acr{xml}-Developer Kit (XDK):  Es bietet eine Reihe von Tools f�r die Schnittstelle von \acr{xml} und Oracle. Es bietet Unterst�tzung f�r C, C++, Java und \acr{plsql}.
	\item XSQL Servlet: Erm�glicht die dynamische Generierung von \acr{xml}-""Dokumenten.
	\item weitere Tools wie beispielsweise Oracle iFS, Oracle InterMedia, JDeveloper.
\end{itemize}

F�r das Projekt wurde nur XDB benutzt. Die �brigen Tools werden deshalb und aus Zeitgr�nden hier nicht n�her betrachtet.

\section{Unterstz�tzung des SQL/XML-Standards}
Wie es der \acr{sql}/\acr{xml}-Standard vorschreibt, lassen sich relationale Daten mit \acr{xml}-Daten kombinieren. Der Datentyp wird bei Oracle allerdings als \texttt{XMLType} bezeichnet und nicht als \texttt{XML} wie im Standard vorgeschrieben \citep{Tuerker2003}.

Der \acr{sql}/\acr{xml}-Standard ist noch nicht vollst�ndig umgesetzt. In Oracle 9i werden nur die folgenden Elemente des Standards unterst�tzt:
\begin{itemize}
	\item \texttt{XMLAgg()}
	\item \texttt{XMLConcat()}
	\item \texttt{XMLElement()}
	\item \texttt{XMLForest()}
\end{itemize}

In sp�teren Versionen ist laut Oracle eine vollst�ndige Unterst�tzung des Standards geplant (Quelle: \citep{url:oracleTechnologyXML}).

Neben dem im \acr{xml}-Standard beschriebenen Funktionen besitzt Oracle weitere Funktionen f�r die \acr{xml}-Unterst�tzung. Dies sind unter anderem:

\begin{itemize}
	\item \texttt{SYS\_XMLGEN()}: Die Funktion ist �hnlich der Funktion \texttt{XMLElement()}. Der einzige Unterschied besteht darin, dass die Funktion nur ein einziges Attribut aufnehmen kann und dieses in \acr{xml} umwandelt.
	\item \texttt{SYS\_XMLLAGG()}: Aggregiert alle Elemente eines \acr{xml}-Dokuments oder ein Fragment davon und erzeugt ein \acr{xml}-Dokument. Das daraus entstehende Dokument wird in einem Element mit dem Defaultnamen "`ROWSET"' eingeschlossen.
\end{itemize}

(Quelle: \citep{OracleSQL})

Es existieren eine Reihe von Methoden des Typs \texttt{XMLType}. Die folgende Liste ist eine unvollst�ndige Aufz�hlung dieser Methoden, die auch im untersuchten Projekt benutzt werden:

\begin{itemize}
  \item \texttt{XMLType()}: Konstruktor zum Erzeugen einer \texttt{XMLType}-Instanz
	\item \texttt{updateXML()}: �nderung eines einzelnen \acr{xml}-Elements.
	\item \texttt{extract()}: Erm�glicht eine Abfrage mit XPath und liefert einen Teilbaum zur�ck.
	\item \texttt{existsNode()}: Pr�ft, ob ein Knoten existiert. Wenn der Knoten vorhanden ist, so wird eine 1 zur�ckgeliefert, andernfalls eine 0.
	\item \texttt{getClobVal()}: Liefert den Wert einer \texttt{XMLType}-Instanz als \acr{clob}-Wert zur�ck.
	\item \texttt{getStringVal()}: Liefert den Wert einer \texttt{XMLType}-Instanz als Zeichenkette zur�ck.
	\item \texttt{getNumVal()}: Liefert den Wert einer \texttt{XMLType}-Instanz als numerischen Wert zur�ck.
\end{itemize}

\section{Unterst�tzung von XPath und XQuery}

Oracle unterst�tzt au�erdem XPath. Dazu stellt das Oracle-\acr{dbms} die Funktion \texttt{extract()} zur Verf�gung.

XQuery hingegen wird erst ab Version 10i weitestgehend unterst�tzt. (Quelle: \citep{url:oracleXQuery})