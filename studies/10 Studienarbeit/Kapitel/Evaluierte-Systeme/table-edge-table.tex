\begin{table}
	\sffamily
	\centering
	\footnotesize
		\begin{tabularx}{\textwidth}{lX}
		\toprule
		
		\multicolumn{1}{@{}N}{Spaltenname} & \multicolumn{1}{V{6em}@{}}{Bedeutung} \\
		%\cmidrule(r){1-1}\cmidrule(l){2-2}		
		\midrule\addlinespace\addlinespace
		
		id					&	bezeichnet eindeutig einen Knoten eines Dokuments. \\																	
		parentid		& verweist auf den Vaterknoten bei einem Element, bei Textknoten \zB auf das zugeh�rige Attribut oder Element.	\\													
		nodetype		& identifiziert den Knotentyp unter Verwendung des \acr{dom}-Nummerierungsschemas f�r Knotenarten.		\\											
		localname		& enth�lt den Element- oder Attributnamen.		\\					
		prefix			& stellt das Namensraumpr�fix dar. \\
		namespaceuri & URI des zugeh�rigen Namensraumes.	\\
		datatype		& gibt den Datentyp von Elementen und Attributen an. \\															
		prev				& enth�lt die Knotennummer des vorhergehenden Geschwisterknoten. \\																		
		text				& nimmt den Attributwert oder Elementinhalt auf. \\																																					
		\addlinespace
		\bottomrule
		
		\end{tabularx}
	\caption{Format der Edge Table, \citep[Kap. 11.3.2]{KlettkeMeyer}}
	\label{tab:edge-table-format}
\end{table}