\begin{table}[H]
	\sffamily
	\centering
	%\footnotesize
	\begin{tabularx}{\textwidth}{llX}
		\toprule
		
		\multicolumn{1}{@{}N}{Kriterium} & 
		\multicolumn{1}{@{}N}{Erf�llt?} & 
		\multicolumn{1}{@{}N}{Beschreibung} \\
		\midrule\addlinespace
		
		Elemente	& \ding{51} & Der zu indexierende Knoten wird �ber einen XPath-Ausdruck ausgew�hlt \\ \cmidrule{1-3}
	  Attribute	& \ding{51} & Der zu indexierende Knoten wird �ber einen XPath-Ausdruck ausgew�hlt \\ \cmidrule{1-3}
	  Text			& \ding{51} & Der zu indexierende Knoten wird �ber einen XPath-Ausdruck ausgew�hlt \\ 
								
		\addlinespace\bottomrule
		
  \end{tabularx}
	\caption{Bewertung: Indexierung}
	\label{tab:oracle-bewertung-indexierung}
\end{table}

Beim Erstellen eines Index gibt Oracle keine Fehlermeldung zur�ck. Wir konnten nicht herausfinden, ob der Zugriff auf die Daten durch setzen eines Index auf ein Attribut oder XML Element tats�chlich beschleunigt wird. Dazu h�tten wir Tests mit gro�en Datenmengen durchf�hren m�ssen, die im Rahmen dieser Studienarbeit nicht m�glich waren. 