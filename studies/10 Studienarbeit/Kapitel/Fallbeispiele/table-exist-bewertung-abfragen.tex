Mit den von eXist zur Verf�gung gestellten Abfragem�glichkeiten (XQuery samt Update-Extensions) lie�en sich die f�r das Fallbeispiel ben�tigten Datenbankanfragen leicht umsetzen. Der Einsatz einer standardisierten Sprache wie XQuery bietet gro�e Vorteile bei der Portabilit�t sowie der Unterst�tzung durch externe Tools (wie \zB XQuery-Debugger oder Visualisierungstools).

Allerdings gibt es auch in eXist eine Reihe von Funktionen, die nicht im XQuery-Standard vorhanden sind; als Beispiel seien die f�r die Durchf�hrung von Updates hinzugef�gten Erweiterungen genannt.

Tabelle \ref{tab:exist-bewertung-abfragen} fasst die in eXist vorhandenen M�glichkeiten zusammen.

\begin{table}
	\sffamily
	\centering
	%\footnotesize
	\begin{tabularx}{\textwidth}{llX}
		\toprule
		
		\multicolumn{1}{@{}N}{Kriterium} & 
		\multicolumn{1}{@{}N}{Erf�llt?} & 
		\multicolumn{1}{@{}N}{Beschreibung} \\
		\midrule\addlinespace
		
		XPath 									& \ding{51} & XPath 2.0, \acr{w3c} Working Draft aus dem November 2003. Standard-Funktionen nicht vollst�ndig implementiert. \\ \cmidrule{1-3}
	  XQuery 									& \ding{51} & XQuery 1.0, \acr{w3c} Working Draft aus dem November 2003. Standard-Funktionen nicht vollst�ndig implementiert. Zus�tzliche Auswahl an eXist-spezifischen Funktionen sowie Erweiterungen f�r die Volltextsuche. \\ \cmidrule{1-3}
	  �nderungsoperationen		& \ding{51}	& Mittels XUpdate bzw. eigenen XQuery-Erweiterungen. �nderungen sind auf Dokumenten- bzw. Node-Ebene m�glich. Bearbeitet werden k�nnen Element-, Attribut und auch Text-Knoten. \\ \cmidrule{1-3}
		Ergebnis-Darstellung		&  					& Mittels der XQuery-Engine lassen sich �ber das \acr{http}-\acr{api} Ergebnisdaten als String bzw. \acr{xml}-String zur�ckgeben. �ber das "`XML:DB"'-\acr{api} kann bei der Verwendung von Java mit enstprechenden Objekten gearbeitet werden. Letztere Zugriffsm�glichkeit (sowie die weiteren, \acr{xml-rpc} sowie \acr{soap}) wurde in unserem Fallbeispiel jedoch nicht verwendet. \\ \cmidrule{1-3}
		Transaktionssicherheit	& \ding{51}	& Im derzeitigen Entwicklungs-Stadium gibt es keine f�r den Benutzer direkt sicht- bzw. nutzbaren Transaktionen. Es werden jedoch intern alle Schreibvorg�nge in Transaktionen gekapselt und alle Operationen in einem Journal aufgezeichnet. Bei einem Datenbankfehler k�nnen so alle durchgef�hrten Transaktionen wiederholt, alle noch nicht durchgef�hrten Transaktionen zur�ckgenommen werden. F�r das Fallbeispiel wurden die entsprechenden Optionen f�r das Logging aktiviert -- es traten jedoch keine Probleme auf, die eine Wiederherstellung n�tig gemacht h�tten. \\
								
		\addlinespace\bottomrule
		
  \end{tabularx}
	\caption{Bewertung: Abfragen}
	\label{tab:exist-bewertung-abfragen}
\end{table}