\begin{table}
	\sffamily
	\centering
	%\footnotesize
	\begin{tabularx}{\textwidth}{llX}
		\toprule
		
		\multicolumn{1}{@{}N}{Kriterium} & 
		\multicolumn{1}{@{}N}{Erf�llt?} & 
		\multicolumn{1}{@{}N}{Beschreibung} \\
		\midrule\addlinespace
		
		Unterst�tzte Speicherformen 			& 					& \acr{xml}-Dokument wird in Baumdarstellung
																										gewandelt, Speicherung mittels B+-B�umen \\ \cmidrule{1-3}
	  Komplette \acr{xml}-Dokumente 		& \ding{51} & einzig sinnvolle Speicherungsform, da bei eXist kein
	  																								"`Tabellen"'-Konzept vorhanden ist \\ \cmidrule{1-3}
	  Tabelle von \acr{xml}-Typ 				& \ding{55} & \textit{nicht zutreffend} \\ \cmidrule{1-3}
		Hybride Speicherung 	& \ding{55} & \textit{nicht zutreffend} \\ \cmidrule{1-3}	  
		\acr{sql}/\acr{xml}-Standard 			& \ding{55} & \textit{nicht zutreffend} \\ 
								
		\addlinespace\bottomrule
		
  \end{tabularx}
	\caption{Bewertung: Speicherung}
	\label{tab:exist-bewertung-speicherung}
\end{table}

Wie Tabelle \ref{tab:exist-bewertung-speicherung} schon andeutet, ist eXist als natives \acr{xml}-\acr{dbms} nicht in jeder Hinsicht mit relationalen \acr{dbms} vergleichbar: Da die Daten nicht in "`Tabellenform"' gespeichert werden, sind die darauf bezogenen Bewertungspunkte nicht anwendbar.
