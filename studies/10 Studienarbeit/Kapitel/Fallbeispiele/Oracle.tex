\section{Einleitung}

Die Umsetzung des Studienprojekts erfolgte auf der Basis von Oracle 9i. Da das Oracle-\acr{dbms} in dieser Version noch kein XQuery unterst�tzt, wurden alle Abfragen in \acr{sql}/\acr{xml} erstellt. Die \acr{sql}/\acr{xml}-Anweisungen werden dar�ber hinaus -- wo n�tig -- mit XPath-Anweisungen kombiniert. 

\section{Erstellen von Indexen}
Bei Oracle l�sst sich ein Index sowohl auf ein Attribut wie auf ein einzelnes Element anwenden. Es lie� sich im Rahmen der Studienarbeit allerdings nicht feststellen, ob damit eine beschleunige Suche m�glich ist.

\lstinputlisting[language=SQLORA,caption={Erstellen von Indexen},label=listing:oracle-index]{Kapitel/Fallbeispiele/oracle-index.txt}

\section{Abfragetypen}
\subsection{Suchen von Ferienwohnungen}

Das folgende Beispiel zeigt die Suche nach einer Wohnung:

\lstinputlisting[language=SQLORA,caption={Insert-Anweisung mit dem Datentyp XMLType in Oracle},label=listing:oracle-xml-suche-wohnung]{Kapitel/Fallbeispiele/oracle-xml-suche-wohnung.txt}

\texttt{XMLELEMENT()} und \texttt{XMLATTRIBUTES()} sind aus dem \acr{sql}-""Standard und erzeugen ein \acr{xml}-""Element bzw. ein \acr{xml}-""Attribut. Die Oracle-eigene Anweisung \texttt{SYS\_XMLAGG()} dient dazu, das umschlie�ende \acr{xml}-Element \texttt{<Wohnungen>} zu erstellen. Standardm��ig besitzt das umschlie�ende Element den Namen \texttt{<ROWSET>}. Um dies zu �ndern, wird \texttt{XMLFORMAT()} benutzt.

Das Vorhandensein einer Ausstattung (wie beispielsweise eines Telefons) wird �ber die XPath-Funktion \texttt{existsNode()} ermittelt. Sie liefert eine 1 zur�ck, wenn -- wie im Beispiel -- das Element \texttt{<Telefon>} vorhanden ist. Andernfalls liefert die Funktion eine 0 zur�ck.

Schlie�lich sorgt die Methode \texttt{getClobVal()} daf�r, dass das resultierende Ergebnis als \acr{clob} zur�ckgegeben wird. 

\subsection{Buchen von Ferienwohnungen}
Bevor die Wohnung gebucht wird, muss zun�chst ermittelt werden, ob die Wohnung im entsprechenden Zeitraum �berhaupt verf�gbar ist. Dies wird mit der folgenden Abfrage durchgef�hrt:

\lstinputlisting[language=SQLORA,caption={Create Table Anweisung mit dem Datentyp XMLType in Oracle},label=listing:oracle-xml-buche-ferienwohnung]{Kapitel/Fallbeispiele/oracle-xml-buche-ferienwohnung.txt}

Mit \texttt{extract()} wird der Attributknoten \texttt{von} bzw. \texttt{bis} ausgew�hlt und mittels der Funktion \texttt{getStringVal()} als String zur�ckgegeben. Diese Umwandlung ist n�tig, damit die Daten mit der Funktion \texttt{TO\_DATE()} in das Format YYYY-MM-DD umgewandelt werden k�nnen. 

Die vom Benutzer in der Applikation eingegebenen Daten werden ebenfalls mithilfe der Funktion \texttt{TO\_DATE()} in das Format YYYY-MM-DD umgewandelt, um die Daten dann anschlie�end im gleichen Format vergleichen zu k�nnen. 

\subsection{Hinzuf�gen von Ferienwohnungen}
%Das folgende Beispiel zeigt die Erstellung der Tabelle Wohnung. Der einzige Unterschied zwischen Standard-\acr{sql} und \acr{sql}/\acr{xml} besteht in der Benutzung des Datentyps \texttt{XMLType}:

%\lstinputlisting[language=SQLORA,caption={Create Table Anweisung mit dem Datentyp XMLType in Oracle},label=listing:oracle-xml-create-tabel]{Kapitel/Fallbeispiele/oracle-xml-create-tabel.txt}

Das folgende Beispiel zeigt einen INSERT in die Tabelle Wohnung. Beim Einf�gen von \acr{xml}-Daten muss das Schl�sselwort \texttt{XMLType} vorangestellt werden. Oracle �berpr�ft beim Einf�gen, ob die Daten wohlgeformt sind:

\lstinputlisting[language=SQLORA,caption={Insert mit XML-Daten},label=listing:oracle-xml-insert]{Kapitel/Fallbeispiele/oracle-xml-insert.txt}

\subsection{�ndern von Ferienwohnungen}

F�r das Update des \acr{xml}-Typs wird die Funktion \texttt{XMLUpdate()} benutzt. Die Funktion verlangt als ersten Parameter eine Instanz des Typs \texttt{XMLType}, als zweiten Parameter einen XPath-String, der das zu �ndernde Element selektiert und als letzten Parameter den zu �ndernden Wert.

\lstinputlisting[language=SQLORA,caption={�ndern einer Ferienwohnung},label=listing:oracle-xml-update]{Kapitel/Fallbeispiele/oracle-xml-update.txt}

\subsection{L�schen von Ferienwohnungen}

Da das L�schen der Daten �ber die ID der Wohnung erfolgt, ist daf�r keine \acr{xml}-Funktionalit�t erforderlich.

\lstinputlisting[language=SQLORA,caption={L�schen einer  Ferienwohnung},label=listing:oracle-xml-delete]{Kapitel/Fallbeispiele/oracle-xml-delete.txt}

\section{Bewertung}
\subsection{Speicherung}
\begin{table}
	\sffamily
	\centering
	%\footnotesize
	\begin{tabularx}{\textwidth}{llX}
		\toprule
		
		\multicolumn{1}{@{}N}{Kriterium} & 
		\multicolumn{1}{@{}N}{Erf�llt?} & 
		\multicolumn{1}{@{}N}{Beschreibung} \\
		\midrule\addlinespace
		
		Unterst�tzte Speicherformen 			& 					& \acr{xml}-Speicherung mit objektrelationalem Typ \texttt{XMLType} oder als \acr{clob} \\ \cmidrule{1-3}
	  Komplette \acr{xml}-Dokumente 		& \ding{51} & ja, aber ohne Vorspann als \texttt{XMLType} \\ \cmidrule{1-3}
	  Tabelle von \acr{xml}-Typ 				& \ding{51} & M�glich mittels \texttt{CREATE TABLE name OF XMLType} \\ \cmidrule{1-3}
		Hybride Speicherung 	& \ding{51} & �ber den Datentyp \texttt{XMLType} m�glich \\ \cmidrule{1-3}	  
		\acr{sql}/\acr{xml}-Standard 			& \ding{51} &   Weitestgehend umgesetzt \\ 
								
		\addlinespace\bottomrule
		
  \end{tabularx}
	\caption{Bewertung: Speicherung}
	\label{tab:oracle-bewertung-speicherung}
\end{table}

Der \acr{sql}/\acr{xml}-Standard ist in Oracle weitestgehend umgesetzt. Bei der Umsetzung der Studienarbeit wurden keine Elemente ben�tigt, die zwar im \acr{sql}-Standard festgelegt sind, aber nicht von Oracle implementiert wurden. 

�ber den \acr{sql}-Standard hinaus bietet Oracle weitere Funktionen an, von denen allerdings im Projekt nur \texttt{SYS\_XMLAGG()} und \texttt{XMLFORMAT()} benutzt wurden. 


\subsection{Abfragen}
Die Umsetzung brachte einige komplexe Abfragen hervor. Vergleicht man die Abfragen mit solchen f�r ein rein relationale \acr{dbms}, f�llt auf jeden Fall eine erh�hte Komplexit�t auf. Dies ist auch darauf zur�ckzuf�hren, dass \acr{xml}-Daten und relationale Daten gemischt wurden.

Tabelle \ref{tab:oracle-bewertung-abfragen} fasst die in Oracle vorhandenen M�glichkeiten zusammen.

\begin{table}
	\sffamily
	\centering
	%\footnotesize
	\begin{tabularx}{\textwidth}{llX}
		\toprule
		
		\multicolumn{1}{@{}N}{Kriterium} & 
		\multicolumn{1}{@{}N}{Erf�llt?} & 
		\multicolumn{1}{@{}N}{Beschreibung} \\
		\midrule\addlinespace
		
		XPath 									& \ding{51} & Wird unterst�tzt durch \texttt{existNode()}, \texttt{extractValue()}, \texttt{extract()} und \texttt{XMLSqeuence()}.  \\ \cmidrule{1-3}
	  XQuery 									& \ding{55} & Wird erst ab Oracle 10i unterst�tzt. \\ \cmidrule{1-3}
	  �nderungsoperationen		& \ding{51}	& Mittels Update der \acr{xml}-Daten durch die Funktion \texttt{UPDATEXML()}. \acr{xml}-Daten werden auf Wohlgeformtheit hin �berpr�ft.  \\ \cmidrule{1-3}
		Ergebnis-Darstellung		&  					& Die Daten lassen sich konvertieren. Mittels \texttt{getClobVal()}, \texttt{getNumVal()} und \texttt{getStringVal()} k�nnen die Daten als CLOB, numerische Wert oder als String zur�ckgegeben werden.  \\ \cmidrule{1-3}
		Transaktionssicherheit	&  & nicht bekannt \\
								
		\addlinespace\bottomrule
		
  \end{tabularx}
	\caption{Bewertung: Abfragen}
	\label{tab:oracle-bewertung-abfragen}
\end{table}

\subsection{Indexierung}
\begin{table}[H]
	\sffamily
	\centering
	%\footnotesize
	\begin{tabularx}{\textwidth}{llX}
		\toprule
		
		\multicolumn{1}{@{}N}{Kriterium} & 
		\multicolumn{1}{@{}N}{Erf�llt?} & 
		\multicolumn{1}{@{}N}{Beschreibung} \\
		\midrule\addlinespace
		
		Elemente	& \ding{51} & Der zu indexierende Knoten wird �ber einen XPath-Ausdruck ausgew�hlt \\ \cmidrule{1-3}
	  Attribute	& \ding{51} & Der zu indexierende Knoten wird �ber einen XPath-Ausdruck ausgew�hlt \\ \cmidrule{1-3}
	  Text			& \ding{51} & Der zu indexierende Knoten wird �ber einen XPath-Ausdruck ausgew�hlt \\ 
								
		\addlinespace\bottomrule
		
  \end{tabularx}
	\caption{Bewertung: Indexierung}
	\label{tab:oracle-bewertung-indexierung}
\end{table}

Beim Erstellen eines Index gibt Oracle keine Fehlermeldung zur�ck. Wir konnten nicht herausfinden, ob der Zugriff auf die Daten durch setzen eines Index auf ein Attribut oder XML Element tats�chlich beschleunigt wird. Dazu h�tten wir Tests mit gro�en Datenmengen durchf�hren m�ssen, die im Rahmen dieser Studienarbeit nicht m�glich waren. 
