Die Umsetzung brachte einige komplexe Abfragen hervor. Vergleicht man die Abfragen mit solchen f�r ein rein relationale \acr{dbms}, f�llt auf jeden Fall eine erh�hte Komplexit�t auf. Dies ist auch darauf zur�ckzuf�hren, dass \acr{xml}-Daten und relationale Daten gemischt wurden.

Tabelle \ref{tab:oracle-bewertung-abfragen} fasst die in Oracle vorhandenen M�glichkeiten zusammen.

\begin{table}
	\sffamily
	\centering
	%\footnotesize
	\begin{tabularx}{\textwidth}{llX}
		\toprule
		
		\multicolumn{1}{@{}N}{Kriterium} & 
		\multicolumn{1}{@{}N}{Erf�llt?} & 
		\multicolumn{1}{@{}N}{Beschreibung} \\
		\midrule\addlinespace
		
		XPath 									& \ding{51} & Wird unterst�tzt durch \texttt{existNode()}, \texttt{extractValue()}, \texttt{extract()} und \texttt{XMLSqeuence()}.  \\ \cmidrule{1-3}
	  XQuery 									& \ding{55} & Wird erst ab Oracle 10i unterst�tzt. \\ \cmidrule{1-3}
	  �nderungsoperationen		& \ding{51}	& Mittels Update der \acr{xml}-Daten durch die Funktion \texttt{UPDATEXML()}. \acr{xml}-Daten werden auf Wohlgeformtheit hin �berpr�ft.  \\ \cmidrule{1-3}
		Ergebnis-Darstellung		&  					& Die Daten lassen sich konvertieren. Mittels \texttt{getClobVal()}, \texttt{getNumVal()} und \texttt{getStringVal()} k�nnen die Daten als CLOB, numerische Wert oder als String zur�ckgegeben werden.  \\ \cmidrule{1-3}
		Transaktionssicherheit	&  & nicht bekannt \\
								
		\addlinespace\bottomrule
		
  \end{tabularx}
	\caption{Bewertung: Abfragen}
	\label{tab:oracle-bewertung-abfragen}
\end{table}