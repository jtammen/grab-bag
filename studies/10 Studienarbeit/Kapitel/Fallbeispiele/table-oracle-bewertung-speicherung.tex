\begin{table}
	\sffamily
	\centering
	%\footnotesize
	\begin{tabularx}{\textwidth}{llX}
		\toprule
		
		\multicolumn{1}{@{}N}{Kriterium} & 
		\multicolumn{1}{@{}N}{Erf�llt?} & 
		\multicolumn{1}{@{}N}{Beschreibung} \\
		\midrule\addlinespace
		
		Unterst�tzte Speicherformen 			& 					& \acr{xml}-Speicherung mit objektrelationalem Typ \texttt{XMLType} oder als \acr{clob} \\ \cmidrule{1-3}
	  Komplette \acr{xml}-Dokumente 		& \ding{51} & ja, aber ohne Vorspann als \texttt{XMLType} \\ \cmidrule{1-3}
	  Tabelle von \acr{xml}-Typ 				& \ding{51} & M�glich mittels \texttt{CREATE TABLE name OF XMLType} \\ \cmidrule{1-3}
		Hybride Speicherung 	& \ding{51} & �ber den Datentyp \texttt{XMLType} m�glich \\ \cmidrule{1-3}	  
		\acr{sql}/\acr{xml}-Standard 			& \ding{51} &   Weitestgehend umgesetzt \\ 
								
		\addlinespace\bottomrule
		
  \end{tabularx}
	\caption{Bewertung: Speicherung}
	\label{tab:oracle-bewertung-speicherung}
\end{table}

Der \acr{sql}/\acr{xml}-Standard ist in Oracle weitestgehend umgesetzt. Bei der Umsetzung der Studienarbeit wurden keine Elemente ben�tigt, die zwar im \acr{sql}-Standard festgelegt sind, aber nicht von Oracle implementiert wurden. 

�ber den \acr{sql}-Standard hinaus bietet Oracle weitere Funktionen an, von denen allerdings im Projekt nur \texttt{SYS\_XMLAGG()} und \texttt{XMLFORMAT()} benutzt wurden. 
