\begin{table}[H]
	\sffamily
	\centering
	%\footnotesize
	\begin{tabularx}{\textwidth}{llX}
		\toprule
		
		\multicolumn{1}{@{}N}{Kriterium} & 
		\multicolumn{1}{@{}N}{Erf�llt?} & 
		\multicolumn{1}{@{}N}{Beschreibung} \\
		\midrule\addlinespace
		
		Elemente	& \ding{51} & Werden im \emph{Struktur-Index} gespeichert, welcher automatisch vom System verwaltet und von den meisten XPath/XQuery-Ausdr�cken verwendet wird.\\ \cmidrule{1-3}
	  Attribute	& \ding{51} & Werden ebenfalls im \emph{Struktur-Index} gespeichert. \\ \cmidrule{1-3}
	  Text			& \ding{51} & Werden im \emph{Volltext-Index} gespeichert. Es werden standardm��ig \textit{alle} Text-Knoten indiziert, was sich jedoch durch eine entsprechende Konfiguration �ndern l�sst. Der Volltext-Index wird von den zus�tzlichen XQuery-Operatoren \texttt{\&=} und \texttt{|=} sowie von den Textfunktionen \texttt{match-all()}, \texttt{near()} usw. verwendet.\\ 
								
		\addlinespace\bottomrule
		
  \end{tabularx}
	\caption{Bewertung: Indexierung}
	\label{tab:exist-bewertung-indexierung}
\end{table}

Tabelle \ref{tab:exist-bewertung-indexierung} zeigt die von eXist indizierten Entit�ten. Zus�tzlich k�nnen manuell \emph{typisierte}, sog. "`Bereichsindexe"' erstellt werden. Deren Verwendung wird in Kapitel \ref{fallbeispiel-exist-indexe} angesprochen.