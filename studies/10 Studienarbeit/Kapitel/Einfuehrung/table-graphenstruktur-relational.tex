\begin{table}[H]
	\sffamily
	\centering
	%\footnotesize
		\begin{tabularx}{\textwidth}{*{5}{|l}|X|}
			\hline
			\multicolumn{1}{|l}{DocID} & \multicolumn{1}{l}{Elementname} & \multicolumn{1}{l}{ID} & \multicolumn{1}{l}{Vorg�nger-ID} & \multicolumn{1}{l}{Kind-Nr} & Wert \\ 
			\hline\addlinespace\hline
			i0001 & item & 101 &  & 1 &  \\ 
			i0001 & title & 102 & 101 & 1 & Benq mit erneutem... \\ 
			i0001 & link & 103 & 101 & 2 &  http://www.heise.de/... \\ 
			i0001 & information & 104 & 101 & 3 &  \\ 
			i0001 & date & 105 & 104 & 1 & 2005-10-15 \\ 
			i0001 & publisher & 106 & 104 & 2 & Peter Muster \\ 
			\hline
		\end{tabularx}
	\caption{Graphenstruktur auf eine relationale Datenbank abgebildet}
	\label{tab:xml-graphenstruktur-rel-datenbank}
\end{table}