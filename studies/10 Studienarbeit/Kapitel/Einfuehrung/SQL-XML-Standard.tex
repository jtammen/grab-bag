Mit der Revision des \acr{sql}-Standards 2003 wurde \acr{sql} unter anderem um \acr{xml}-F�higkeiten erweitert. Diese Erweiterung wird h�ufig auch als \acr{sql}/\acr{xml} bezeichnet. Im Gegensatz zu z.B. XPath wurde \acr{sql}/\acr{xml} \emph{nicht} vom \acr{w3c} standardisiert, sondern von der ISO/ANSI. 

Die Erweiterungen des Standards erlaubt es, \acr{xml}-Daten in einer relationalen Datenbank abzulegen oder aus Daten \acr{xml}-Dokumente zu generieren. 

\section{Der Datentyp XML}
Mit \acr{sql}/\acr{xml} wurde der neue Datentyp \texttt{XML} eingef�hrt, der in der Lage ist, \acr{xml}-Daten aufzunehmen \citep{Tuerker2003}.  Das folgende Beispiel zeigt eine Tabelle "`Lieferant"', die ein Feld "`Bemerkungen"' besitzt, in dem \acr{xml}-Daten eingef�gt werden k�nnen:

\lstinputlisting[language=SQL,caption={Create Table mit Datentyp XML},label=listing:xml-create-table,morekeywords={XML}]{Kapitel/Einfuehrung/CreateTable.txt}

Der Datentyp \texttt{XML} erlaubt die Speicherung folgender Daten:
\begin{itemize}
	\item \texttt{NULL}
	\item \acr{xml}-Inhalte
	\item \acr{xml}-Dokumente (\acr{xml}-Daten mit Vorspann)
\end{itemize}

\section{XML-Funktionen} \label{einf�hrung-sqlxml-funktionen}
\acr{sql} wurde um die folgenden \acr{xml}-Funktionalit�ten erweitert:
\begin{itemize}
	\item \texttt{XMLGEN}: Erm�glicht das Einf�gen von XQuery-Abfragen.
	\item \texttt{XMLELEMENT}: Erzeugt ein \acr{xml}-Element
	\item \texttt{XMLATTRIBUTES}: Erzeugt ein \acr{xml}-Attribut
	\item \texttt{XMLFOREST}: Erzeugt einen Wald von \acr{xml}-Elementen
	\item \texttt{XMLCONCAT}: konkateniert \acr{xml}-Elemente
	\item \texttt{XMLAGG}: Funktion zum Aggregieren oder gruppieren von \acr{xml}-Werten. 
\end{itemize}

Hinweis: Alle \texttt{XML}-Funktionen -- au�er XMLAGG -- k�nnen �ber \texttt{XMLGEN} und XQuery nachgebildet werden.

\section{Abbildung zwischen SQL und XML}
\acr{sql} und \acr{xml} basieren auf unterschiedlichen Datenmodellen. Daher muss das \acr{sql}-Datenmodell auf das \acr{xml}-Datenmodell abgebildet werden und umgekehrt. Diese Abbildung zeigt Tabelle \ref{tab:abbildung-sql-xml}.

\begin{table}
	\sffamily
	\centering
	
	\begin{tabularx}{\textwidth}{lX}
		\toprule
		
		\multicolumn{1}{@{}N}{\acr{sql}-Datenmodell} & 	 \multicolumn{1}{V{6em}@{}}{\acr{xml}-Datenmodell} \\
		\midrule\addlinespace\addlinespace

		\acr{sql}-Bezeichner 		& \acr{xml}-Name \\ %\cmidrule{1-2}
		\acr{sql}-Zeichens�tze 	& \acr{xml}-Unicode \\ %\cmidrule{1-2}
		\acr{sql}-Datentypen 		& \acr{xml}-Schema-Typen \\ %\cmidrule{1-2}
		\acr{sql}-Tabelle 			& \acr{xml}-Dokument \\ %\cmidrule{1-2}
		\acr{sql}-Schemata 			& \acr{xml}-Dokument + \acr{xml}-Schema-Dokument \\ %\cmidrule{1-2}
		\acr{sql}-Katalog 			& \acr{xml}-Dokument + \acr{xml}-Schema-Dokument \\ 
		\addlinespace
		\bottomrule
		\end{tabularx}
	\caption{Abbildung \acr{sql} $\leftrightarrow$ \acr{xml}}
	\label{tab:abbildung-sql-xml}
\end{table}

Tabelle \ref{tab:abbildung-sql-typen-xml-typen} zeigt die Abbildung von \acr{sql}-Datentypen in \acr{xml}-Schema-Typen.

\begin{table}[H]
	\sffamily
	\centering
	
	\begin{tabularx}{\textwidth}{lX}
		\toprule
		
		\multicolumn{1}{@{}N}{\acr{sql}-Datentyp} & 	 \multicolumn{1}{V{6em}@{}}{\acr{xml}-Schema-Typ} \\
		\midrule\addlinespace\addlinespace

		\texttt{BOOLEAN} & boolean \\ %\cmidrule{1-2}
		\texttt{CHAR, VARCHAR, CLOB} & string \\ %\cmidrule{1-2}
		\texttt{BIT, VARING BIT, BLOB} & base64Binary, hexBinary \\ %\cmidrule{1-2}
		\texttt{SMALLINT, INT, BIGINT} &  integer \\ %\cmidrule{1-2}
		\texttt{NUMERIC, DECIMAL} & decimal \\   %\cmidrule{1-2}
		\texttt{FLOAT, REAL, DOUBLE PRECISION} & float, double \\ %\cmidrule{1-2}
		\texttt{DATE} & date \\ %\cmidrule{1-2}
		\texttt{TIME} & time \\ %\cmidrule{1-2}
		\texttt{TIMESTAMP} & dateTime \\ %\cmidrule{1-2}
		\texttt{INTERVAL} & duration \\
		\addlinespace
		\bottomrule
		\end{tabularx}
	\caption{Abbildung \acr{sql}-Datentypen $\leftrightarrow$ \acr{xml}-Schema-Typen}
	\label{tab:abbildung-sql-typen-xml-typen}
\end{table}