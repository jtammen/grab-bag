\begin{table}
	\sffamily
	\centering
	\footnotesize
		\begin{tabularx}{\textwidth}{llX}
		\toprule
		
		\multicolumn{1}{@{}N}{} & \multicolumn{1}{@{}N}{\acr{xml}-Information} & \multicolumn{1}{V{6em}@{}}{Datenbank-Informationen} \\
		%\cmidrule(r){1-1}\cmidrule(l){2-2}		
		\midrule\addlinespace\addlinespace
		
		Element						&	\acr{xml}-Element																& Attribut einer Relation \\
											& Sequenz von Elementen														&	Attribute einer Relation \\
											& Alternative	von Elementen												& Attribute einer Relation \\
											& Element mit dem Quantifizierer "`?"' 						& Attribut, Nullwerte m�glich \\
											& Elemente mit Quantifizierer "`+"' oder "`*"' 		& SET oder LIST \\
											& komplex stukrutiertes (geschachteltes) Element 	& ROW Type \\
		Attribut					& \acr{xml}-Attribut															& Attribut einer Relation \\
											& \#IMPLIED																				& Nullwert erlaubt \\
											& \#REQUIRED																			& Nullwert nicht erlaubt \\
											& Defaultwert																			& Defaultwert	\\			
		\addlinespace
		\bottomrule
		
		\end{tabularx}
	\caption{Datenbankstruktur f�r ein DTD-Schema, \citep[Kap. 8.4.1]{KlettkeMeyer}}
	\label{tab:xml-datenbankstruktur-automatisch}
\end{table}