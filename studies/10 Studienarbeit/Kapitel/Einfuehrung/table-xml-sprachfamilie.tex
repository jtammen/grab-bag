\begin{table}
	\sffamily
	\centering
	%\footnotesize
	\begin{tabularx}{\textwidth}{lX}
		\toprule
		
		\multicolumn{1}{@{}N}{Bezeichnung} & \multicolumn{1}{V{6em}@{}}{Beschreibung} \\
		%\cmidrule(r){1-1}\cmidrule(l){2-2}		
		\midrule\addlinespace\addlinespace
		
		\acr{xsl} 			& Sprachfamilie zur Layout-Erzeugung f�r \acr{xml}-Dokumente. Zu \acr{xsl} geh�ren \acr{xslt} und \acr{xsl-fo} \\ \cmidrule{1-2}
		\acr{xslt} 			& Geh�rt zur \acr{xsl}-Familie und dient zur Transformation von Dokumenten in andere Formate. �ber \acr{xslt}-Stylesheets l�sst sich \zB ein \acr{xml}-Dokument in ein \acr{html}- bzw. \acr{xhtml}-Dokument transferieren, um es �ber einen Webbrowser anzuzeigen.\\ \cmidrule{1-2}
		XLink 					& Syntax zur Definition von bi- oder auch multidirektionalen, externen Links in \acr{xml}.\\ \cmidrule{1-2}
		XPointer 				& Auch "`\acr{xml} Pointer Language"'. Anfragesprache, um mittels XLink-Ausdr�cken bzw. in URIs auf einzelne Teile eines \acr{xml}-Dokumentes zu verweisen.\\ \cmidrule{1-2}
		XML Namespaces 	& �ber Namensr�ume soll die Eindeutigkeit von \acr{xml}-Elementen sichergestellt und somit Namenskollisionen verhindert werden. \\ \cmidrule{1-2}
		XML Signature 	& Schreibweise f�r das Signieren von \acr{xml}-Dokumenten. \\ \cmidrule{1-2}
		XML-Encryption 	& Spezifikation f�r das Ver- und Entschl�sseln von \acr{xml}-""Dokumenten. \\ \cmidrule{1-2}
		\acr{xsd} 			& Siehe Kapitel \ref{einfuehrung-xml-xsd}. \\ \cmidrule{1-2}
		\acr{xhtml} 		& Auf \acr{xml} basierender Nachfolger von \acr{html}. \\ \cmidrule{1-2}
		\acr{svg} 			& Erm�glicht die Beschreibung von 2D--Vektor Grafiken in \acr{xml}. \\ \cmidrule{1-2}
		\acr{soap} 			& �ber dieses Protokoll k�nnen Daten zwischen Systemen ausgetauscht, sowie \acr{rpc}s durchgef�hrt werden. \\ \cmidrule{1-2}
		\acr{rdf} 			& Dient zur Repr�sentation von Metadaten. \\ \cmidrule{1-2}
		XPath 					& Siehe Kapitel \ref{einfuehrung-xml-xpath}. \\ \cmidrule{1-2}
		XQuery 					& Siehe Kapitel \ref{einfuehrung-xml-xquery}. \\
								
		\addlinespace
		\bottomrule
		
		\end{tabularx}
	\caption{Einige XML-Architekturstandards}
	\label{tab:xml-standards}
\end{table}