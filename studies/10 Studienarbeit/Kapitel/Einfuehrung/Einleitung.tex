\chapter{Einleitung}
\section{Zielsetzung der Arbeit}
Da die Verwendung von \acr{xml} als Datenaustauschformat in den letzten Jahren stark zugenommen hat, gibt es auch einen steigenden Bedarf an Speicherungsm�glichkeiten von \acr{xml}-Daten in herk�mmlichen, relationalen Datenbanken. Einige Anbieter von \acr{dbms} bieten deshalb spezielle Erweiterungen ihrer Systeme an -- au�erdem wird versucht, den \acr{sql}-Standard konsequenterweise um \acr{xml}-Eigenschaften zu erweitern.

In der vorliegenden Arbeit soll untersucht werden, welche \acr{xml}-Funktionalit�ten heutige (relationale) \acr{dbms} bieten und inwiefern die Verwendung dieser Funktionen Vorteile \zB bez�glich des Austauschs von \acr{xml}-Dokumenten mit sich bringen. Hierbei ist besonders von Interesse, ob und wie sich bei der Verwendung von \acr{xml} die an das Datenbanksystem zu stellenden Anfragen �ndern und ob alle in \acr{sql} vorhandenen M�glichkeiten abgebildet werden k�nnen.
Da sich inzwischen auch eine gr��ere Anzahl nativer \acr{xml}-\acr{dbms} auf dem Markt platziert hat, gehen wir auch auf die Verwendung eines solchen alternativen Konzepts ein.

\section{Vorgehensweise}
Zur Vorbereitung des Lesers seien zun�chst die relevanten Technologien und Sprachen rund um \acr{xml} genannt und vorgestellt. Anschlie�end werfen wir einen Blick auf die aktuelle Revision des \acr{sql}-Standards, welcher von der \acr{iso} verabschiedet wurde und u.a. einige �nderungen in Bezug auf \acr{xml} enth�lt.

Weiterhin werden die verschiedenen Techniken beleuchtet, mit denen eine dauerhafte Speicherung von \acr{xml}-Dokumenten erreicht werden kann -- im Allgemeinen kann hier zwischen "`hybriden"' und "`nativen"' Methoden unterschieden werden.

Nach Vermittlung dieser technischen Voraussetzungen stellen wir die in dieser Arbeit zu evaluierenden \acr{dbms} vor. Die o.g. Fragen sollen anschlie�end anhand einer Beispiel-Applikation untersucht werden. Da verst�ndlicherweise nicht alle existenten \acr{dbms} auf ihre \acr{xml}-Funktionalit�t behandelt werden k�nnen, beschr�nken wir uns dabei auf zwei der etabliertesten \acr{dbms} (Oracle sowie Microsoft SQL-Server), sowie eine kleineres, natives \acr{xml}-\acr{dbms} (eXist).

Im letzten Teil der Arbeit werden die in Bezug auf die Beispiel-Applikation ermittelten Ergebnisse vorgestellt und die Funktionsweisen der einzelnen \acr{dbms} miteinander verglichen.