Abschlie�end k�nnen wir feststellen, dass die \acr{xml}-Unterst�tzung der untersuchten \acr{dbms} sehr umfangreich ist und viele M�glichkeiten f�r die Verarbeitung von \acr{xml}-Daten bietet. Die f�r das Anwendungsbeispiel aufgestellten Anforderungen konnten mit allen drei \acr{dbms} umgesetzt werden.

Auf der anderen Seite ist die breite Unterst�tzung von \acr{xml}-Funktionalit�ten auch ein Nachteil, da sich der Einarbeitungsaufwand dadurch erh�ht. Bei den "`\acr{xml}-enabled"' Datenbanken steigert sich die Komplexit�t der Anfragen durch Benutzung der \acr{xml}-Funktionen erwartungsgem��.

Die Umsetzung des \acr{sql}/\acr{xml}-Standards ist sowohl bei Oracle 9i wie auch beim SQL-Server 2005 nicht vollst�ndig, wobei Oracle sich wesentlich n�her am Standard orientiert. XPath und XQuery sind beim SQL-Server und eXist vollst�ndiger umgesetzt als bei Oracle. 

Unserer Ansicht nach war der gr��te Nachteil bei der Umsetzung der Anwendung, dass bisher keine Modellierungssprache bei der Verwendung von \acr{xml} existiert. Folglich wird ein strukturiertes Vorgehen bei der Analyse deutlich erschwert. Dieses Problem tritt besonders bei der hybriden Speicherung von \acr{xml}-Daten auf. Wir haben das Problem dadurch umgangen, indem wir uns am bestehenden relationalen Datenmodell orientiert haben.

In unserem Fallbeispiel wurden durch die Benutzung von \acr{xml} keine M�glichkeiten geschaffen, die nicht auch durch die Benutzung von rein relationalen Daten bestanden h�tten. Vorteile h�tten sich ergeben, wenn ein Datenaustausch zwischen den \acr{dbms} existiert h�tte. Beispielsweise wenn die Wohnungsdaten bereits in \acr{xml} vorgelegen h�tten oder ein \acr{xml}-Export der Rechnungsdaten n�tig gewesen w�ren.