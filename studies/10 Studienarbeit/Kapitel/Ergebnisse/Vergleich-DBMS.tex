\section{Allgemeines}
\subsection{Speicherung}
Bei Oracle werden \acr{xml}-Fragmente als "`XMLType"' gespeichert und bei dem SQL-Server als "`XML"'-Datentyp. Bei beiden k�nnen vollst�ndige Dokumente eingelesen werden, jedoch wird der Vorspann entfernt. Die Kommentare bleiben bei beiden erhalten.
Im Gegensatz zu Oracle bietet der SQL-Server kein objekt-relationales Datenmodell. Bei Oracle lassen sich, da es sich beim Datentyp "`XMLType"' um einen objekt-relationalen Typ handelt, auch Tabellen dieses Typs anlegen.
Beide \acr{dbms} unterst�tzen die kombinierte Speicherung von relationalen und \acr{xml}-Daten. 
Die Speicherungstechnik von eXist l�sst sich mit den von relationalen \acr{dbms} nicht direkt vergleichen, da es speziell f�r die effiziente Speicherung und Abfrage von \acr{xml}-Daten entwickelt wurde.

W�hrend Oracle die im \acr{sql}-\acr{xml}-Standard definierten Funktionen (s. Kapitel \ref{einf�hrung-sqlxml-funktionen}) sowie den Datentyp "`XML"' gr��tenteils umgesetzt hat, arbeitet Microsoft mit eigenen, �hnliche Funktionalit�t bietenden Operationen und Datentyp.

\subsection{Abfragen}
Zur Abfrage von \acr{xml}-Daten bieten die drei \acr{dbms} die Abfragesprache XPath an. XQuery hingegen wird (mit Einschr�nkungen im Funktionsumfang) nur vom SQL-Server 2005 sowie von eXist angeboten; Oracle hingegen unterst�tzt XQuery erst ab Version 10g, welche f�r das Fallbeispiel nicht zur Verf�gung stand.

Zur Bearbeitung von \acr{xml}-Daten stellen Oracle und SQL-Server eigene Funktionen zur Verf�gung. Es lassen sich damit Updates auf Attribut-, Element- und Textebene durchf�hren. Die \acr{xml}-Daten werden in beiden \acr{dbms} als Stringtyp zur�ckgegeben -- datenbankintern k�nnen sie in andere Datentypen konvertiert werden, \zB um Datumsfunktionen zu benutzen.
Die Transaktionssicherheit kann bei Oracle und SQL-Server auf Spalten angewendet werden, nicht auf Teilbereiche von XML-Daten.

Zur Durchf�hrung von �nderungsoperationen implementiert eXist einige XQuery-Erweiterungen, mit denen sich ebenfalls Attribut-, Element- und Textknoten bearbeiten lassen. Im Fallbeispiel werden die Anfrageergebnisse von eXist mittels XQuery als \acr{xml}-String zur�ckgegeben.

\subsection{Indexierung}

Wie in Kap. \ref{sql-server-xml-index} beschrieben, unterst�tzt der SQL-Server die Indexierung des gesamten \acr{xml}-Dokumentes und spezifisch auf Pfade, Element- und Attributwerte sowie f�r Operationen auf die Werte.

Standardm��ig werden bei eXist alle Textknoten automatisch indiziert, zus�tzlich kann man benutzerdefinierte Indexe erstellen.

In Oracle lassen sich manuell Indexe auf Elemente, Attribute und Textknoten setzen. 

\section{Gegen�berstellung einiger Abfragen}
\subsection{Suchen von Ferienwohnungen}

\lstinputlisting[language=SQLORA,caption={Suchen von Ferienwohnungen, Oracle},label=listing:vergleich-ferienwohnung-suchen-oracle]{Kapitel/Fallbeispiele/oracle-xml-suche-wohnung.txt}

\lstinputlisting[language=SQL05,caption={Suchen von Ferienwohnungen, SQL-Server},label=listing:vergleich-ferienwohnung-suchen-sql-server]{Kapitel/Fallbeispiele/sql-server-wohnung-suchen.txt}

\lstinputlisting[language=XQuery,caption={Suchen von Ferienwohnungen, eXist},label=listing:vergleich-ferienwohnung-suchen-exist]{Kapitel/Fallbeispiele/exist-suche-naiv.xql}

In der Formulierung der Abfragebedingungen unterscheiden sich die drei Beispiele kaum; lediglich bei der Zusammenstellung der Ausgabe gibt es gro�e Differenzen. Das Ausgabeergebnis zu formulieren ist bei eXist mittels XQuery deutlich weniger aufwendig.

\subsection{Hinzuf�gen von Ferienwohnungen}
\lstinputlisting[language=SQLORA,caption={Hinzuf�gen von Ferienwohnungen, Oracle},label=listing:vergleich-ferienwohnung-hinzufuegen-oracle]{Kapitel/Fallbeispiele/oracle-xml-insert.txt}

\lstinputlisting[language=SQL05,caption={Hinzuf�gen von Ferienwohnungen, SQL-Server},label=listing:vergleich-ferienwohnung-hinzufuegen-sql-server]{Kapitel/Fallbeispiele/sql-server-wohnung-einfuegen.txt}

Wie aus den obigen Beispielen ersichtlich ist, unterscheiden sich die Einf�geoperationen allein durch den Aufruf des \acr{xml}-Konstruktors bei Oracle. Der \acr{sql}-Server nimmt den Typ implizit an.

\lstinputlisting[language={[eXist]XQuery},caption={Hinzuf�gen von Ferienwohnungen, eXist},label=listing:vergleich-ferienwohnung-hinzufuegen-exist]{Kapitel/Fallbeispiele/exist-einfuegen.xql}

