Neben der Implementierung der Basisumgebung für Oz bietet das Mozart Programming
System auch so genannte \textsl{System Module}, die ein effektiveres und
effizienteres Entwickeln von Anwendungen ermöglichen. Die Module lassen sich in
mehrere Einsatzgebiete gliedern, die im Folgenden erläutert werden. 
 
\section{Application Programming}

Für das \textsl{Application Programming} finden sich zwei Module, die das
Arbeiten mit Applikationen und das Verwenden der vorhandenen Module erleichtern.

\begin{description}
  \item[\texttt{Application}] Das Modul \texttt{Application} ermöglicht den
  Zugriff auf die Argumente einer Anwendung und bietet Prozeduren um
  Anwendungen zu beenden. Der Zugriff auf die Argumente ist vergleichbar mit
  \texttt{Strings[] args} in Java. Es stehen 3 verschiedene Argumentarten für
  Web-, Kommandozeilen- und GUI-Anwendungen zur Verfügung. Außerdem können mit
  Hilfe dieses Moduls die Argumente auf unterschiedliche Arten geparst und
  verarbeitet werden. 
  \item[\texttt{Module}] Für den Zugriff, das Linken und Laden von Modulen ist
  das Modul \texttt{Module} zuständig. Hier kann auf vorhandene Module, wie z.B.
  \texttt{Application}, zugriffen werden um Prozeduren zu verwenden. Außerdem
  können eigene Oz-Anwendungen als Module geladen werden und später dann gelinkt
  werden.
\end{description}

\section{Constraint Programming}

Die Module, die in den Bereich des \textsl{Constraint Programming} fallen,
stellen einen der wichtigsten Teile für die Erweiterung von Oz dar. Hier finden
sich Module, die die Semantik und Syntax für das Constraint Programming
erweitern. Beispielsweise wird die Semantik durch das Modul \texttt{FD} um
\textsl{Finite Set Constraints} erweitert, eine Erweiterung auf Mengen von
Constraints. 

Ein kurzer Überblick aller Module für das Constraint Programming,
\begin{description}
  \item[\texttt{Search}] Dieses Modul bietet Unterstützung für die \textsl{Suche} von
  Lösungen, näheres siehe Punkt \ref{subsection:Search}. 
  
  \item[\texttt{FD}] Für den leichteren Umgang mit \textit{Finite Domain Constraints}
  finden sich in diesem Module beispielsweise vordefinierte Propagatoren, siehe
  Punkt \ref{subsection:FD}.
  
  \item[\texttt{Schedule}] Das Modul \texttt{Schedule} unterstützt die
  Entwicklung von Scheduling-Anwendung\-en, in denen Tasks auf gegebene Ressourcen
  verteilt werden, so dass keine zeitlichen Überlappungen auftreten. In
  \cite[System Modules, Part II]{url:mozart-documentation} wird diese
  Vermeidung von zeitlichen
  Überlappungen als \textsl{Serialization for Unary Resources} bezeichnet.
  Außerdem bietet das Modul Propagatoren und Verteiler um mehrere Tasks, die
  verschiedene Ressourcen benötigen, so zu verteilen, damit jede Ressource
  \textsl{serialized} ist. Zusätzlich wird auch noch das so genannte kumulative
  Scheduling unterstützt, hier dürfen die Kapazitäten der Ressourcen nicht
  überschritten werden.
  
  \item[\texttt{FS}] Hinter diesem Modul verbergen sich \textsl{Finite Set
  Constraints}. Es handelt sich um eine neue Constraint-Art für Oz, die eine
  $n$-elementige Menge mit $n$ finite Domain Variablen assoziiert. Mengen von
  Constraint Variablen erweisen sich, laut \cite[System Modules, Part
  II]{url:mozart-documentation}, bei
  kombinatorischer Suche und Natural Language Processing als sehr nützlich. Im
  Umfang des Moduls befinden sich Prozeduren, Propagatoren etc. für den
  einfachen Umgang mit Finite Set Constraints.
  
  \item[\texttt{RecordC}] Um auch mit anderen Datentypen als Domain Variablen,
  Intervall von Integern, auf den Constraint Store zuzugreifen, bietet das
  Modul \texttt{RecordC} die Möglichkeit, Constraints vom Datentyp
  \texttt{Record} zu verwenden. Es werden Prozeduren angeboten um das
  \textsl{Telling} auf den Store für Records zu erweitern.
  
  \item[\texttt{Combinator}] Damit Constraints miteinander kombiniert werden
  können, bietet dieses Modul Prozeduren zur Kombination der Constraints. Bei den
  Kombinatoren handelt es sich hauptsächlich um konditionale oder logische, wie
  z.B. \texttt{Combinator.'or'} oder \texttt{Combinator.'not'}. Hier handelt
  sich um eine direkte Erweiterung der Oz-Syntax, die ohne weiteren Aufwand
  verwendet werden kann, daher werden die Operatoren auch als String aufgerufen.
  
  \item[\texttt{Space}] Mit Hilfe dieses Moduls können eigene \textsl{Inferenzmaschinen}
  erstellt werden. Es werden Prozeduren bereitgestellt, mit denen Spaces
  verwaltet werden können. Durch die eigene Verwaltung, wie das Erzeugen neues
  Spaces \texttt{Spaces.create} oder das Zusammenfügen von Spaces
  \texttt{Spaces.merge}, können für die Inferenzmaschine neue Schlussfolgerungen
  erreicht werden. 
\end{description}

Im Folgenden wird näher auf die Module \texttt{Search} und \texttt{FD} 
eingegangen, da diese den meisten Einsatz beim Constraint Programming finden.

\subsection{Modul \texttt{Search}}
\label{subsection:Search}

Für die Suche bietet das Modul 3 Möglichkeiten, die
\textsl{Basis-Suchmaschinen}, die \textsl{Universellen Suchmaschinen} und die
\textsl{Parallele Suche}. Allgemein ist zur Suche in Mozart zu sagen, dass nach
einer bzw. der ersten möglichen Lösung, allen Lösungen und der besten Lösung
gesucht werden kann. Für die Suche nach der besten Lösung muss noch eine Ordnung
übergeben werden, die definiert, was eine gute oder schlechte Lösung ist.

\subsubsection{Basis-Suchmaschinen}

Um die Suche mit den Basis-Suchmaschinen zu ermöglichen, müssen in den
Prozeduren, mit denen eine Lösung gefunden werden soll, Entscheidungspunkte 
gesetzt werden. Dieses Setzten der Punkte erfolgt über das Schlüsselwort
\texttt{choice}, das nebenbei bemerkt aus dem Modul \texttt{Combinator} stammt. 

Ein kleines Beispiel nach \cite{LogicProgr:2003} zur Verdeutlichung. Hier sollen
nach den Kinder eines gegebenen Vaters gesucht werden.

\lstinputlisting[
	mathescape=false,
	caption={Suche nach allen Lösungen}, 
	label={lst:baseSearch}]
	{../baseSearch.oz} 

Der letzte Aufruf in Listing \ref{lst:baseSearch} stößt die Suche mit dem Vater
"`terach"' an. Durch den Aufruf \texttt{Search.base.all} wird nach allen
möglichen Lösungen gesucht und als Ergebnis wird die Liste \texttt{[abraham
nachor haran]} zurückgegeben.

\subsubsection{Universelle Suchmaschinen}

Mit den Universellen Suchmaschinen lässt sich eine parametrisierte Suche
durchführen. Hierzu bietet das Modul die Möglichkeit des so genannten
\textsl{Recomputation} \cite[System Modules, Part II, Chapter
4.2]{url:mozart-documentation}, die Suche zu stoppen
und verschiedene Ausgabemodi zu verwenden.

Mit Hilfe des Recomputation können auch für sehr große Suchbäume, für die in der
Regel nicht genügend Speicherplatz zur Verfügung steht, Lösungen gefunden
werden. Bei der normalen Suche in Oz wird in jedem Distributionschritt eine
neue Kopie des Spaces erstellt. Dadurch kann der Speicherbedarf bei sehr großen
Suchbäumen schnell sehr groß werden. Um diesem Problem Abhilfe zu schaffen, kann
mittels der Recomputation angegeben werden, dass nur alle $n$ Distributionsschritte
eine Kopie erstellt werden soll. Die Verringerung des Speicherbedarfs geht
allerdings auf die Kosten der Zeit. Schlägt eine Suche im $n-1$-ten Schritt nach
der letzten Kopie fehl, muss zur letzten Kopie zurück gesprungen werden und
einige Suchpfadteile könnten mehrfach durchsucht werden.

\subsubsection{Parallele Suche}

Mit Hilfe der parallelen Suche soll die Exploration von Suchbäumen beschleunigt
werden, siehe \cite[System Modules, Part II, Chapter
4.4]{url:mozart-documentation}. Um die Suche zu parallelisieren,
wird der Suchbaum in mehrer Untersuchbäume aufgeteilt und auf mehrere
Oz-Prozesse verteilt. Die Verteilung auf mehrere Oz-Prozesse kann lokal auf
einen PC laufen oder auch auf mehrere PCs im Netzwerk verteilt werden. Damit die
Suche parallel laufen kann, stellt das Modul die Klasse \texttt{Search.parallel}
bereit, die alle Aufgaben übernimmt.

\subsection{Finite Domain Constraints: \texttt{FD}}
\label{subsection:FD}

Ein weiterer wichtiger Teil für das Constraint Programming stellt das Modul
\texttt{FD} für \textsl{Finite Domain Constraints} dar. Das Modul definiert
Prozeduren, Propagatoren und Distributions-Strategien für die Suche.

Die meisten hier definierten Prozeduren und Propagatoren können implizit in Oz
verwendet werden. Beispielsweise bedarf es bei Prozeduren für das Telling auf den
Constraintstore keinen expliziten Import des Moduls. Die definierten
Propagatoren gliedern sich in generische, symbolische, und binäre (0/1
Propagatoren), vgl. \cite[System Modules, Part II, Chapter 5
]{url:mozart-documentation}. Die generischen
Propagatoren lassen Definitionen für den Wertebereich von Domain Variablen zu,
z.B. kleiner gleich "`\texttt{=<:}"', die auch implizit in Oz verwendet werden
können. 

Zusätzlich bietet das Modul \textsl{Reflection} und das so genannte
\textsl{Watching}, nach \cite[System Modules, Part II, Chapter
5.5]{url:mozart-documentation}, für Domains (Domain
Variablen) an. Bei Reflection kann z.B. die aktuelle untere oder obere Grenze
der Domain Variable zurückgegeben werden, \texttt{reflect.min} bzw.
\texttt{reflect. max}. Mittels des Watching können Domains beobachtet werden.
Beispielsweise wird \texttt{true} bei \texttt{\{FD.watch.size *Domain1 +Domain2
?B\}} zurück gegeben, wenn die Intervallgröße von \texttt{Domain1} kleiner
wird als die Größe von \texttt{Domain2}.

\section{Verteilte Programmierung}

Für die Entwicklung von \textsl{verteilten Anwendungen} bietet das Mozart
Programming System einen Satz von Modulen, die die Entwicklung erleichtern.
Neben Modulen für den Verbindungsaufbau oder die Remote-Steuerung von
Oz-Prozessen, finden sich auch Module für die Manipulation für URLs (Uniform
Resource Locator) oder für statistische Analysen der Verteilung.

Ein Überblick aller verfügbaren Module für die verteilte Programmierung, 

\begin{description}
  \item[\texttt{Connection}] Um in Oz eine Verbindung zwischen Oz-Prozessen
  herzustellen, werden so genannte \textsl{Tickets}
  \cite[System Modules, Part III]{url:mozart-documentation} verwendet. Ein
  Ticket stellt einen String
  dar, der für den Verbindungsaufbau verschickt wird. Um eine Verbindung
  herzustellen, erzeugt ein Prozess, der so genannte \textsl{Server}, ein Ticket,
  das vom \textsl{Client} auf der anderen Seite geholt bzw. empfangen wird.
  Das Modul \texttt{Connection} unterscheidet hier zwei Verbindungstypen,
  nach \cite[System Modules, Part III]{url:mozart-documentation}:
  \begin{description}
    \item[One-to-one] Erlaubt nur Einzelverbindungen zwischen 2 Prozessen, mit
    einem einmal verwendbaren Ticket.
    \item[Many-to-one] Erlaubt Verbindung von mehreren Clients zu einem Server
    mit dem gleichen Ticket.
  \end{description}
  \item[\texttt{Remote}] Mittels dieses Moduls können über das Netzwerk andere
  Oz-Prozesse gestartet werden. Diese neu erzeugten Prozesse können
  \textsl{remote} gesteuert werden. Von der Klasse \texttt{Remote.manager} wird
  auf einem Computer im Netzwerk eine Instanz erzeugt, der so genannte
  \textsl{Remote Modul Manager}, der es erlaubt remote den Prozess zu steuern
  und auf die Ressourcen des entfernten Computers zuzugreifen. 
  \item[\texttt{URL}] Das Modul \texttt{URL} dient zur Erzeugung und Manipulation
  von URLs, für das WWW und das Dateisystem. Durch die Prozedur \texttt{URL.make}
  wird eine URL erzeugt, die sich als Datenstruktur verwalten lässt,
  beispielsweise \texttt{\{URL.make 
  "'http://www. mozart\-oz.org/home-1.1.0/share/FD.ozf"'\}}. Das Modul kümmert
  sich um die systemspezifischen Seperatoren für Verzeichnisse, somit spielt es
  keine Rolle, ob Verzeichnisse durch \verb+\+ oder \verb+/+ getrennt werden.
  
  \item[\texttt{Resolve}] Damit schnell Dateien oder Unterverzeichnisse in
  einer URL gefunden und verarbeitet werden können, bietet das Modul
  \texttt{Resolve} einige Prozeduren. Wie der Name schon sagt, soll die
  übergebene URL nach verschiedenen Arten aufgelöst werden, z.B. kann mit
  \texttt{root} das Rootverzeichnis für alle übergebenen URLs gesetzt werden.
  
  \item[\texttt{Fault}] Mit Hilfe dieses Moduls können Fehler bei Verteilung
  erkannt und behandelt werden. Anzumerken ist an dieser Stelle, dass dieses
  Modul laut der Dokumentation \cite[System Modules, Part
  III]{url:mozart-documentation} im aktuellen
  Release (Version 1.3.2) noch nicht komplett fertig gestellt ist.
  
  \item[\texttt{Discovery}] Bei der Lokation von Diensten (Oz-Servern) im
  Netzwerk hilft das Modul \texttt{Discovery}. Ein Oz-Server wird im einen Wert
  initialisiert, in der Regel mit einen Ticket, und wartet bis ein Client eine
  Anfrage startet. Sucht ein Client einen Dienst im Netzwerk sendet er einen
  Broadcast über das Netzwerk und wartet auf die Antwort des zugehörigen
  Servers. In der aktuellen Version 1.3.2 ist die Funktionalität nur für Linux-
  und Solaris-Netzwerke verfügbar, siehe \cite[System Modules, Part
  III]{url:mozart-documentation}. 
  
  \item[\texttt{DPInint}] Die so genannte Verteilungsschicht, nach
  \cite[System Modules, Part III]{url:mozart-documentation}, wird bei Mozart nur
  bei Bedarf dynamisch
  nachgeladen. Über das Modul \texttt{DPInit} können einige Parameter zur
  Ladezeit der Verteilungsschicht initialisiert werden. \texttt{IpPort} legt
  beispielsweise fest über welchen Port kommuniziert werden soll. 
  
  \item[\texttt{DPStatistics}] Das Modul \texttt{DPStatistics} unterstützt die
  statistische Analyse von verteilten Anwendungen, um möglicherweise einige
  Aspekte verbessern zu können. Mit Hilfe des Moduls können Logfiles in
  verschiedenen Ausgabemodi erstellt werden. 
\end{description}

\section{Open Programming}

In der Mozart Dokumentation \cite[System Modules, Part
IV]{url:mozart-documentation}  werden diese Module
als Verbindung zum Rest der \textsl{Computational World} beschrieben. Mit Hilfe
dieser Module erhält man Zugriff auf Dateien und Zugang zum Betriebssystem.

\begin{description}
  \item[\texttt{Open}] Mit diesem Modul ermöglicht Mozart den Zugriff auf
  Dateien (\texttt{Open.file}), Sockets (\texttt{Open.socket}) und Pipes
  (\texttt{Open.pipe}). Wie üblich können Dateien geöffnet, eingelesen,
  geschrieben und wieder geschlossen werden. Mit Sockets können Verbindungen zum
  Netzwerk über TCP oder UDP geöffnet und verarbeitet werden. Über die Klasse
  \texttt{Open.pipe} können Prozesse über die übergebene PID (Process ID)
  \textsl{geforkt} werden, um Daten mit diesem Prozess auszutauschen.
  Zusätzlich beinhaltet das Modul die Klasse \texttt{Open.text}, die in
  Verbindung mit den anderen Klassen verwendet werden kann um Text buchstaben-
  oder zeilenweise einzulesen. Außerdem werden Exceptions von diesem Modul
  ausgelöst, falls Probleme beim Zugriff auftreten, z.B. \textsl{already closed}.
  
  \item[\texttt{OS}] Die Prozeduren des Moduls \texttt{OS} sind laut
  \cite[System Modules, Part IV]{url:mozart-documentation} zu POSIX (Portable
  Operating System Interface)
  kompatibel. Hauptsächlich handelt es sich um Prozeduren für die Interaktion
  mit dem Betriebssystem, z.B. das Erzeugen von Zufallszahlen (\texttt{OS.rand})
  oder Zugriff auf das Dateisystem (\texttt{OS.tmpnam}, absoluter Pfad zu einer
  neu erzeugten Datei). Außerdem erhält man auch Zugriff auf Verzeichnisse,
  beispielsweise liefert \texttt{OS.getDir} alle Pfade zu Dateien im gegebenen
  Verzeichnis. Zusätzlich bietet das Modul Zugriff auf Sockets, die Systemzeit und
  Umgebungsvariablen. Bei den so genannten \textsl{Low Level Procedures} können
  Angaben über die Lese- oder Schreibberechtigung gemacht werden, z.B. read only
  usw. Zu diesen Prozeduren zählen auch Prozeduren, wie \texttt{OS.kill}, für
  die Prozess-Steuerung, näheres siehe \cite[System Modules, Part IV]{url:mozart-documentation}.
\end{description}

\section{System Programmierung}

In das Gebiet der System Programmierung fallen Module, die die
\textsl{Funktionalität} das Mozart Programming System betreffen. Im Folgendem
einen Überblick dieser Module,

\begin{description}
  \item[\texttt{Pickle}] Das Modul \texttt{Pickle} dient dazu, zustandslose
  Werte zu speichern, also um Peristenz herzustellen. Die Werte können unter
  einem Dateinamen gespeichert und über diesen wieder geladen werden, z.B.
  komprimierte Speicherung \texttt{Pickle.save\-Compressed}.
  
  \item[\texttt{Property}] Um die Parameter der \textsl{Mozart Engine} (Virtual
  Maschine) abzufragen und zu aktualisieren, dient dieses Modul. Hierzu wird
  eine Schnittstelle mit 3 Prozeduren bereitgestellt. Zum Setzten von
  \textsl{Properties} die Prozedur \texttt{Property.put} und für den Zugriff
  auf die Parameter \texttt{Property.get} (nur Lesen des Parameters) und
  \texttt{Property.condGet}. Die Einstellungen können beispielsweise die
  Distribuierung, die Spaces oder den Speichernutzung betreffen, näheres siehe
  \cite[System Modules, Part V]{url:mozart-documentation}.
  
  \item[\texttt{Error}] Damit Fehlermeldungen, in Form von Exceptions, leserlich
  gestaltet werden können, bietet das Modul \texttt{Error} die Möglichkeit die
  Fehler zu formatieren. So genannte \textsl{Error Formatters} können erstellt
  werden, die Aussage über einen speziellen Fehler geben können. Außerdem bietet
  das Modul Prozeduren zur Fehlerbehandlung in einen Fehlerfall, z.B.
  \texttt{Error.printException}. 
  
  \item[\texttt{ErrorFormatters}] Mozart stellt im Modul
  \texttt{ErrorFormatters} schon einige vordefinierte Fehlermeldungen bereit,
  die von \texttt{Error}-Modul verwendet werden und auch selbst verwendet
  werden können. Beispielsweise formatiert \texttt{os} Exceptions und Fehler,
  die von der Schnittstelle zum Betriebssystem hoch kommen.  
  
  \item[\texttt{Finalize}] Mit Hilfe dieses Moduls können in Functors oder
  Klassen gekapselte Daten (Ressourcen) automatisch wieder freigegeben werden,
  falls der Functor oder die Klasse nicht mehr existiert. Hierfür wird in
  \cite[System Modules, Part V]{url:mozart-documentation} ein Beispiel für eine
  Datenbankverbindung, die
  in einer Klasse gekapselt ist, vorgestellt. Falls die Klasse nicht mehr
  benötigt wird, wird die Instanz durch die Garbage Collection zerstört. Bevor
  allerdings die Instanz gelöscht wird, sollte die Datenbankverbindung
  geschlossen werden, diese Aufgabe übernimmt das Modul \texttt{Finalize}.
  
  \item[\texttt{System}] Im Modul \texttt{System} werden Prozeduren
  zusammengefasst, die in Verbindung mit der Mozart Engine stehen. Die meisten
  Prozeduren beziehen sich auf die formatierte Ausgabe von beispielsweise
  Fehler oder einfachen Strings.
\end{description} 

\section{Window Programming}

Wie bei jeder höheren Programmiersprache stellt Mozart Oz Module für die
Oberflächenprogrammierung bereit. Zur Programmierung nutzt Mozart eine
objektorientierte Schnittstelle zu \textsl{Tk} (Toolkit). Ursprünglich entstand
Tk mit der Skriptsprache Tcl \footnote{\url{http://www.tcl.tk/}} (Tool Command
Language) und stellt ein plattformunabhängiges \textsl{Graphical User Interface
Toolkit} dar, siehe \cite{url:tcl-tk}.  

\begin{description}
  \item[\texttt{Tk}] Dieses Modul stellt die Schnittstelle zu Tk dar. Hier
  können Fenster (\texttt{Tk.frame}), so genannte \textsl{Widgets} oder Buttons
  (\texttt{Tk.button}) zu einer Benutzeroberfläche zusammengestellt werden.
  Außerdem können Bilder zu Darstellung geladen werden. Zusätzlich bietet das Modul
  noch so genannte \textsl{Listeners}, die den Listener des  
  Java-Oberflächenkonzepts ähneln, um die Oberfläche über bestimmte Ereignisse
  zu benachrichtigen, siehe \cite[System Modules, Part VI]{url:mozart-documentation}. 
  
  \item[\texttt{TkTools}] Um die Erstellung von Oberflächen zu erleichtern, wird
  das Modul \texttt{TkTools} von Mozart bereitgestellt. In diesem Modul finden
  sich vordefinierte Oberflächenkomponenten, wie Dialoge
  (\texttt{TkTools.dialog}) oder Menüleisten (\texttt{TkTools.menubar}), die
  einfach in der GUI verwendet werden können.
\end{description}

