%\subsection{Überblick}
\begin{frame}{System Module - Überblick} 
  \begin{itemize}
    \item Erweiterung der Oz Base Environment
    \item Ermöglichen effektiveres und effizienteres Entwickeln
    \item Module für verschiedene Einsatzgebiete
    \item Sozusagen die Standardbibliothek von Oz
  \end{itemize}
\end{frame}
 
\subsection{Application Programming} 
\begin{frame}{Application Programming}
  \begin{itemize} 
    \item 2 Module für Applikationen und das Laden von Modulen
    \item Module
    \begin{description}
      \item[\texttt{Application}] Zugriff auf die Argumente einer
      Applikation
      \begin{itemize} 
  	    \item Vergleichbar mit \texttt{String args[]} in Java
  	    \item Parsen der übergebenen Argumente
	  \end{itemize}
      \item[\texttt{Module}]Laden von Modulen
    \end{description}
  \end{itemize}
\end{frame}
 
\subsection{Constraint Programming}
\begin{frame}{Constraint Programming}
  \begin{itemize}
    \item Erweiterungen für Constraint Programming
    \item Insgesamt 7 Module, die Semantik und Syntax erweitern
    \item Module
    \begin{description}
      \item[\texttt{Search}] Unterstützung von verschiedenen Suchmaschinen
      \item[\texttt{FD}] Erweiterungen für finite domain Constraints
      \item[\texttt{Schedule}] Unterstützung für Scheduling-Anwendungen
      \item[\texttt{FS}] Erweiterung von finite Domain Constraints auf Mengen
      \item[\texttt{RecordC}] Records als Constraints
      \item[\texttt{Combinator}] Kombinatoren für Constraints, z.B. \texttt{or}
      \item[\texttt{Space}] Erweiterung für die Spaces
    \end{description}
  \end{itemize}
\end{frame}

\begin{frame}{Modul \texttt{Search}}
  \begin{itemize}
    \item \textbf{Basis Suchmaschinen} - Suche nach einer Lösung, allen Lösungen oder
    den besten Lösungen.
    \item \textbf{Universelle Suchmaschinen} - Parameterisierte Suche
    \begin{itemize}
      \item Recomputation - Reduzierung des Suchraums (Space)
      \item Beenden der Suche
      \item Verschiedene Ausgabe-Modi
    \end{itemize}
    \item \textbf{Parallele Suche} - Verteilung der Suche auf Rechnern im Netzwerk
    \begin{itemize}
      \item Verteilung durch Aufteilung des Suchbaum in Untersuchbäume
      \item Nur von Vorteil, wenn Suchbaum groß genug und gut unterteilbar
    \end{itemize}
  \end{itemize}
\end{frame}

\begin{frame}{Finite Domain Constraints: \texttt{FD} }
  \begin{itemize}
    \item Erweiterung des Telling auf den Constraint Store für
     Datenstrukturen
    \item Reflection für Domains
    \item Verschiedene vordefinierte Propagatoren, z.B. Generic Propagators
    oder 0/1 Propagators (binär) 
  \end{itemize}
\end{frame}

\begin{frame}{Scheduling Unterstützung: \texttt{Schedule}}
    \begin{itemize}
      \item Propagatoren und Verteiler für Scheduling-Anwendungen
      \item \textit{Serialization for unary resources} - Anordnung von Tasks, die
      gleiche Ressource benötigen, ohne zeitliche Überlappung
      \item Verteilung der Tasks, so dass jede benötigte Ressource \textit{serialized}
      ist
      \item Kumulatives Scheduling - Kapazität einer Ressource darf nicht
      überschritten werden
    \end{itemize} 
\end{frame}

\begin{frame}{Finite Set Constraints: \texttt{FS}}
  \begin{itemize}
    \item Neue Constraint-Art
    \item Assoziation einer $n$-elementigen Menge mit $n$ finite Domain
      Variablen
    \item Mengen von Constraint Variablen sehr nützlich bei kombinatorischer
      Problemlösung und Natural Language Processing
    \item Prozeduren, Propagator usw. für den Umgang mit Finite Set Constraints 
  \end{itemize}
\end{frame}

\begin{frame}{First-class Computation Spaces: \texttt{Space}}
  \begin{itemize}
    \item Zur Programmierung von Interferenzmaschinen 
    \item Prozeduren um mit Computation Spaces umzugehen
  \end{itemize} 
\end{frame}



\subsection{Verteilte Programmierung}
\begin{frame}{Verteilte Programmierung}
  \begin{itemize}
    \item Zur Entwicklung von verteilten Anwendungen
    \item Module
    \begin{description}
      \item[\texttt{Connection}] Ticket-basierter Verbindungsaufbau zwischen unabhängigen Oz
      Prozessen, auch lokal. Ticket ist ein String, der übertragen wird.
      \begin{itemize}
        \item One-to-one - Einzelverbindungen
        \item Many-to-one - Mehrere Verbindungen mit gleichem Ticket
      \end{itemize}
      \item[\texttt{Remote}] Remotesteuerung anderer Oz Prozesse über das
      Netzwerk. 
      \item[\texttt{URL}] Erzeugen und Manipulieren von URLs, für das WWW und
      Dateisystem
    \end{description}
  \end{itemize}
\end{frame} 

\begin{frame}{Verteilte Programmierung}
  \begin{itemize}
    \item Weitere Module:
    \begin{description}
      \item[\texttt{Resolve}] Auflösen von URLs zur einfacheren Auffindung von Daten
      \item[\texttt{Fault}] Erkennung und Behandlung von Fehlern in der Verteilung
      \item[\texttt{Discovery}] Lokation von Diensten (Oz-Server) im Netzwerk
      \item[\texttt{DPInint}] Initialisierung und Konfiguration der Verteilungsschicht
      \item[\texttt{DPStatistics}] Abfrage von statistischen Informationen
    \end{description}
  \end{itemize}
\end{frame}

\subsection{Open Programming}
\begin{frame}{Open Programming}
  \begin{itemize}
    \item Verbindung zum Rest der \textit{computational world}
    \item Module
    \begin{description}
      \item[\texttt{Open}] Zugriff auf Dateien, Sockets und Pipes 
      \item[\texttt{OS}] Support für Betriebssysteme. Prozeduren zur
      Interaktion mit dem Betriebssystem, z.B. Exceptions falls Fehler auftretten
    \end{description}
  \end{itemize}
\end{frame}

\subsection{System Programmierung}
\begin{frame}{System Programmierung}
  \begin{itemize}
    \item Weitere Funktionalität für die Mozart Engine
    \item Module
    \begin{description}
      \item[\texttt{Pickle}] Persistente Speicherung von zustandlosen Werten
      \item[\texttt{Property}] Operationen auf Eigenschaften (Properties)
      \item[\texttt{Error}] Formatierung von Fehlermeldungen, z.B. Exceptions
      \item[\texttt{ErrorFormatters}] Vordefinierte Fehlerformatierungen, z.B.
      \texttt{os} Fehler des Betriebssystems
      \item[\texttt{Finalize}] Automatische Freigabe von Resourcen, bei
      gekapselten Daten
      \item[\texttt{System}] Prozeduren für die Mozart Engine, z.B. Drucken
    \end{description} 
  \end{itemize}
\end{frame}

\subsection{Window Programming}
\begin{frame}{Window Programming}
  \begin{itemize}
    \item Programmierung von Benutzeroberflächen mit Tk
    \item Module
    \begin{description}
      \item[\texttt{Tk}] Modul zur GUI Programmierung. Verschiedene Klassen für
      die Kommunikation zwischen der Graphic Engine und der Mozart Engine
      \item[\texttt{TkTools}] Graphische Tools, vordefinierte Klassen für
      Dialoge, Menüs usw. 
    \end{description}
  \end{itemize}
\end{frame}