\begin{frame}{Mozart - Einleitung}
  \begin{itemize}
    \item Mozart implementiert die multiparadigmische Programmiersprache Oz
    \item Forschungsprojekt seit ca.\ 1995 (B, D, S)
    \item Oz ist plattformunabhängig (Byte-Code ähnlich wie Java)
    \item Open-Source-Lizenz
  \end{itemize}
\end{frame}

\begin{frame}{Mozart - Features}
  \begin{itemize}
    \item Unterstützung mehrerer Programmierparadigmen
    \item Nebenläufigkeit (Threads, Synchronisation)
    \item Transparente Verteilung
    \item Dynamische Typisierung
  \end{itemize}
\end{frame}

\begin{frame}{Oz - Programmiermodell (OPM)}
  \begin{itemize}
    \item Berechnungen laufen in sog.\ "`Spaces"' (Berechnungsraum) ab
    \item \textsl{Aktoren} kommunizieren darin über einen gemeinsamen 
    \textsl{Speicher}, können Information ablegen und abrufen
    \item Grundlage von Oz: Concurrent Constraint Programming, Erweiterungen 
    durch \textsl{syntactic sugar}
  \end{itemize}
\end{frame}

\begin{frame}{Oz - Datentypen}
  \begin{itemize}
    \item Basisdatentypen: \texttt{Number: Int, Float, Char}
    \item Zusammengesetzte Typen: \texttt{Record: Tuple, Literal, Bool}
    \item \texttt{Chunk}: erlaubt Erstellung abstrakter Datentypen, vordefiniert
    z.\,B.\ \texttt{Array, Dictionary, Class}
    \item \texttt{Thread, Space, ByteString, \ldots}
  \end{itemize}
\end{frame}